\documentclass[a4paper,11pt]{article}

\usepackage{cmap}		
%\usepackage[utf8]{inputenc}			
\usepackage[brazil]{babel}
\usepackage{framed}
\usepackage{hyperref}
\usepackage{amsmath}
\usepackage{graphicx}
\setlength {\marginparwidth }{2cm}
\usepackage[colorinlistoftodos]{todonotes}
\usepackage{wrapfig}
\usepackage{lipsum}
\usepackage{listings}
\usepackage{color}
\usepackage{indentfirst}
\usepackage{times}
\usepackage{textcomp}
\usepackage{pgfgantt}
\usepackage{lipsum}
\usepackage{xcolor}
\usepackage{appendix}
% set document font, font sizes, margin dimensions and spacing
\usepackage{fontspec}
\setmainfont{Arial}
\usepackage[top=15mm,bottom=25mm,left=20mm,right=20mm]{geometry}
\usepackage{setspace}\onehalfspacing
\usepackage{titlesec}
\titleformat*{\section}{\Large\bfseries}
\titleformat*{\subsection}{\Large\bfseries}
\titleformat*{\subsubsection}{\Large\bfseries}
\titleformat*{\paragraph}{\Large\bfseries}
\titleformat*{\subparagraph}{\Large\bfseries}
\setlength{\parskip}{0.6em}

\newif\ifblackandwhite
\blackandwhitetrue

\usepackage{etoolbox}
\usepackage{longtable}%
\AtBeginEnvironment{longtable}{%
  \addfontfeature{RawFeature=+tnum;-onum}%  <--- requires LuaTeX
}

\usepackage{pdflscape}
%\usepackage[svgnames]{xcolor}
 \usepackage{colortbl}%
   \newcommand{\myrowcolour}{\rowcolor[gray]{0.925}}
\usepackage{booktabs}

\usepackage{caption}
\usepackage{graphicx}
\usepackage{colortbl}
\usepackage{longtable}

\ifblackandwhite
  \newcommand{\cheading}[2]{\textbf{#1\hfill #2}}
  \newcommand{\highest}[1]{\textbf{#1}}% == highest score for question
\else
  \newcommand{\cheading}[2]{\textcolor{Maroon}{\textbf{#1\hfill #2}}}
  \newcommand{\highest}[1]{\textcolor{Maroon}{\textbf{#1}}}%
\fi

\definecolor{mygray}{rgb}{0.4,0.4,0.4}
\definecolor{mygreen}{rgb}{0,0.8,0.6}
\definecolor{myorange}{rgb}{1.0,0.4,0}

\lstdefinestyle{customc}{
  belowcaptionskip=1\baselineskip,
  breaklines=true,
  frame=L,
  xleftmargin=\parindent,
  language=C,
  showstringspaces=false,
  basicstyle=\footnotesize\ttfamily,
  keywordstyle=\bfseries\color{green!40!black},
  commentstyle=\itshape\color{purple!40!black},
  identifierstyle=\color{blue},
  stringstyle=\color{orange},
  numbers=left,
  numbersep=12pt,
  numberstyle=\small\color{mygray},
}
\lstset{escapechar=@,style=customc}

\newcommand{\HRule}{\rule{\linewidth}{0.5mm}}

%Definindo um comando todoin que aceita quebra de linha e fórmulas
\newcommand\todoin[2][]{\todo[inline, caption={2do}, #1]{
\begin{minipage}{\textwidth-4pt}#2\end{minipage}}}

\newcommand\todogeg[2][]{\todo[inline, caption={#2}, color=yellow!100, #1]{
\begin{minipage}{\textwidth-4pt}#2\end{minipage}}}

\newcommand\todovwcm[2][]{\todo[inline, caption={#2}, color=red!100, #1]{
\begin{minipage}{\textwidth-4pt}#2\end{minipage}}}

\begin{document}

\pagestyle {empty}
\begin{titlepage}
\begin{center}

% logo
%\includegraphics[width=0.35\textwidth]{images/ufrpe_logo_preto.png}~

\begin{figure}[ht]
		\includegraphics[height=3cm]{images/logo_ufrpe_horizontal.png}
		\hspace{3.5cm}
    	\includegraphics[height=2.5cm]{images/logo_bsi .pdf}
	\end{figure}   

\vspace{1cm}

\textsc{\large <Título do Relatório do Aluno>} \\
\small \textcolor{violet}{e.g. Construção de uma API REST \textit{On Premise} utilizando Python e Django para monitoramento de ativos da empresa XPTO}\\[2cm]

% identificação do relatório
\HRule \\[0.4cm]
{\large \bfseries Relatório Técnico relativo ao Trabalho de Conclusão Curso \\
do Bacharelado em Sistemas de Informação na modalidade Empresa \\[0.4cm]}
\HRule 
\\[2cm]

% identificação do aluno
\large\textbf{Aluno}\\
<Nome do Aluno>\\[1cm]

% identificação do orientador
\large\textbf{Orientador}\\
<Nome do Orientador>\\
<Departamento/Unidade Acadêmica>\\[1cm]



\vfill

% Bottom of the page
{\large \today}
\end{center}

%

%folha rosto
{\center
{\large Nome do Aluno}\\[3cm]
{\Large \bf Título do Trabalho}\\[2.0cm]


{\raggedleft
\begin{minipage}[t]{8.3cm}
\setlength{\baselineskip}{0.25in}
Relatório Técnico apresentado ao Curso de Bacharelado em Sistemas de Informação da Universidade Federal Rural de Pernambuco, como requisito parcial para obtenção do título de Bacharel em Sistemas de Informação.
\end{minipage}\\[2cm]}

\begin{center}
	Universidade Federal Rural de Pernambuco -- UFRPE
  	\par
  	Departamento de Estatística e Informática
    \par
  	Curso de Bacharelado em Sistemas de Informação
\end{center}

\vspace{3cm}
{\large \bf{Orientador:} ...}}\\[2.0cm]

\vspace{2cm}
{\center Recife \\[2mm]
Dezembro  de 2021 \\}






\end{titlepage}
\newpage
\begin{abstract}
no máximo 1 página

\end{abstract}

\newpage
\tableofcontents
\newpage
\pagestyle {plain}
\setcounter{page}{0} \pagenumbering{arabic}

\newpage
\section{Introdução}
\label{sec:introducao}

\textcolor{violet}{Nesta seção você deve falar em  qual contexto seu trabalho está inserido. 
Qual o problema que está sendo abordado e qual a proposta de solução (em linhas gerais). 
Qual a motivação/justificativa para o desenvolvimento da solução proposta (Dentre as possibilidades de solução porque essa foi a escolhida). O texto pode ser dividido em subseções.} 
\subsection{Contexto}
\subsection{Problema e solução proposta}
\subsection{Justificativa}

Exemplo de referência para figura

Como visto na Figura~\ref{fig:lifecycle_phase} $\ldots$

\begin{figure}[ht]
    \center
    \includegraphics[scale=0.7]{images/lifecycle_phase.png}
    \caption{Testing levels based on software development phase (Sommerville, 2011).}
    \label{fig:lifecycle_phase}
\end{figure}

Exemplo de referência para bibliografia

De acordo com \cite{sommerville2011software} $\ldots$

\section{A empresa e sua atuação }
\label{sec:atuacao}

{O Centro de Estudos e Sistemas Avançados do Recife, CESAR, foi fundado em 1996 com o proposito de aproximar a academia ao mercado, promovendo iniciativas conjuntas para o desenvolvimento de novos produtos e serviços. No ano 2000 ele foi para o Porto Digital e nos anos seguintes só prosperou. O CESAR está presente dentro do Brasil, tendo Recife como sede e filiais em Manaus, Rio de Janeiro, Sorocaba e Curitiba e também nos Estados Unidos, particularmente na Flórida, onde fica o escritório comercial para contemplar clientes e projetos norte-americanos. 

Como lema "Pessoas impulsionando inovação e inovação impulsionando negócios", a organização conta com um time diverso e multidisciplinar de mais de mil e trezentos colaboradores, incluindo designers, desenvolvedores, consultores, estrategistas, empreendedores, pesquisadores e educadores. Em 2022, o CESAR obteve o certificado Great Place to Work, que trata-se de uma consultoria global que apoia organizações a obter melhores resultados por meio de uma cultura de confiança, alto desempenho e inovação. Essa certificação se reflete em outros números: eNPS de 91.3\%\, que se traduz como índice de satisfação dos colaboradores.

Mais de duas décadas depois da sua criação, a relevância dessa organização se traduz em números: no ano de 2021, o CESAR obteve uma receita de R\$\ 306 milhões, um crescimento de cerca de 60\%\ em relação aos R\$\ 186 milhões de 2020. No ano de 2021 foram mais de 140 projetos executados, mais de 20 segmentos atendidos, mais de 80 clientes atendidos por ano.

Atualmente o CESAR atua em todo o ciclo de inovação, desde pesquisa, formação de pessoas, experimentação e criação de novos modelos de negócios, até o desenvolvimento de soluções robustas e escaláveis. Como principais diferenciais a organização conta com uma escola de inovação, a CESAR School; Uso de Design-Driven, onde todas as soluções criadas são centradas nas pessoas através de abordagens de experimentação; Vasta experiência no desenvolvimento de soluções de alta performance o que garante entregas de classe global; Rede de inovação aberta, onde sempre é buscado parcerias com academia, mercado, empresas maduras e startups. 

Durante esses 25 anos muitas empresas de grande porte buscaram o CESAR para firmaram parcerias e desenvolverem soluções dentre eles: HP, Motorola, Grupo Boticário, Fundação Telefônica Vivo, Petrobras, Globo, Grupo Neoenergia e etc. 

A atuação da autora desse estudo consiste em construir soluções de interface que melhorem a experiência do usuário e converter tais soluções em softwares. Para criar esses softwares aplicando essas soluções são utilizadas tecnologias de front-end. 
}
\section{Desenvolvimento realizado na empresa}
\label{sec:desenvolvimento}

{Esta seção apresenta a ferramenta desenvolvida como solução ao problema de acessibilidade em  \textit{e-commerce} da organização, abordando as técnicas e tecnologias utilizadas e contribuições para a empresa.}

\subsection{A problemática e a solução proposta}

{A problemática tratava-se de transformar um \textit{e-commerce} legado sem práticas de acessibilidade em um \textit{e-commerce} acessível e bem construído focando em dar suporte diminuir as barreiras de acesso que as pessoas com deficiências visuais se deparam ao acessar um sistema como esse. Como requisitos principais para a solução ser aceita se fazia necessário construir ou adaptar os componentes incluindo práticas de acessibilidade, dar suporte a diversos tipos de navegações que o usuário pudesse escolher, construir um painel administrativo onde apresentasse um \textit{dashboard} com gráficos acessíveis para o acompanhamento das vendas.

Para tratar essa problemática foi necessário dividi-la em pequenas fases e trabalhar isoladamente em cada uma delas, visto que essas fases dependiam uma da outra e elas foram desenvolvidas de forma cronológica partindo de pesquisas a testes. 

\begin{figure}[ht]
  	\centering
    \includegraphics[width=0.8\textwidth]{images/guideline-4.png}
    \caption{Fases do Guideline}
    \caption*{Fonte: a autora (2022)}
    \label{fig:guideline}
\end{figure}  

No total foram preciso quatro fases, como exemplificado na figura  \ref{fig:guideline}, sendo a primeira fase voltada para o entendimento do problema e do uso das novas tecnologias como também entender as limitações do projeto. A segunda fase estava voltada na construção e adaptação da interface do usuário seguindo tudo o que foi levantado na fase de entendimento do problema. A terceira fase teve foco na construção do painel administrativo e no \textit{dashboard}, além do planejamento de componentes e na construção de gráficos acessíveis. Por fim a quarta fase, onde se concentrou os testes, foram construídas abordagens de testes para a certificação de que todos os requisitos estavam sendo satisfeitos e se as diretrizes estavam sendo seguidas.

}

\subsection{Fase 1: Pesquisa e entendimento}
{
Para começar a construção do \textit{e-commerce} se fazia necessário entender melhor o problema que iria ser enfrentado e as tecnologias usadas a fim de entender as limitações do projeto em questão de performance. Com o entendimento das limitações do projeto os desenvolvedores teriam a visão da possibilidade ou não de adicionar novas bibliotecas ao decorrer do desenvolvimento caso fosse necessário. Na figura \ref{fig:guideline-f1} é destrinchado o que compreende a fase 1 na construção do \textit{guideline}.

\newpage


\begin{figure}[ht]
  	\centering
    \includegraphics[width=0.75\textwidth]{images/guideline-f1.png}
    \caption{Fase 1 do Guideline}
    \caption*{Fonte: a autora (2022)}
    \label{fig:guideline-f1}
\end{figure} 


Após concluída a fase de estudo e exploração surgiu a necessidade da construção de um guia de acessibilidade que seria usado na fase de desenvolvimento do projeto, essa decisão se deu porque a documentação das diretrizes de acessibilidade se mostrou bastante extensa e englobava muitos tipos de deficiência, tornando o entendimento muito dificultoso para todos que estão lendo. Para complementar o guia de acessibilidade foi construído um \textit{style guide} para guiar todos os desenvolvedores a respeito de aspectos importantes que devem estar presentes na \textit{User Interface (UI)}}

\subsubsection{Guia de acessibilidade}
{
Ao se tocar no quesito cessibilidade é necessário citar a \textit{WCAG} que trata-se de um conjunto de diretrizes que estipula os padrões de acessibilidade digital que devem ser seguidos pelos desenvolvedores para a produção de aplicações \textit{on-line}. Essas recomendações foram todas desenvolvidas pelo consórcio \textit{World Wide Web} (W3C) \cite{W3C}, .

O WCAG é responsável por orientar de forma muito simples como identificar e implementar técnicas que eliminam barreiras de acesso para pessoas com deficiência e ele foi construído sob quatro princípios:
\begin{itemize}
\item Perceptível: As informações e os componentes da interface do usuário devem ser apresentados em formas que possam ser percebidas pelo usuário.
\item Operável: Os componentes da interface e a navegação devem ser operáveis, ou seja o usuário não pode ter impedimentos para utilizar a interface.
\item Compreensível: A informação e a operação da interface de usuário devem ser compreensíveis, elas devem fazer sentido e bem apresentadas na interface do usuário. 
\item Robusto: O conteúdo deve ser robusto o suficiente para poder ser interpretado de forma confiável por uma ampla variedade de agentes de usuário, incluindo tecnologias assistivas. A aplicação deve ser bem construída, de forma a ser acessível para uma gama maior de navegadores e tecnologias assistiva.
\end{itemize}

Guiado por esses quatro princípios surgiram as diretrizes, que são os itens que fornecem os objetivos básicos que devem ser atingidos dentro de cada principio, que são tópicos menores e mais específicos. Dentro das diretrizes existe critérios de sucesso, esse sucesso está atrelado a nível de conformidade. Então quanto mais conforme o \textit{e-commerce} está adequado àquela diretriz maior é o critério de sucesso. Os critérios são definidos três níveis de conformidade:
\begin{itemize}
    \item No nível A estão os critérios mais simples, que representam apenas barreiras mais significativas de acessibilidade. Ao obter sucesso nesse critério já é possível eliminar uma boa quantidade de barreiras de acesso. \ref{Diretrizes nível A}
    \item No nível AA, é um nível acima do A, o que significa que atingindo esse nível as barreiras do nível A foram somadas a mais barreiras e todas foram eliminadas. \ref{Diretrizes nível AA}
    \item Nível AAA, geralmente são um refinamento das anteriores, sendo mais detalhadas e que trazem um nível mais sofisticado de acessibilidade.  \ref{Diretrizes nível AAA}
\end{itemize}

Dentre as 78 diretrizes WCAG \cite{WCAG21} que existem foram escolhidas algumas que fizessem sentido no contexto de diminuição de barreiras de acesso para deficientes visuais. Toda a construção do \textit{e-commerce} foi baseada nas diretrizes listadas no Apêndice \ref{sec:apendice}: 

}

\newpage
\subsubsection{Style Guide}
{O \textit{style guide} \cite{lynch2016web} é um documento que concentra as diretrizes de design de um projeto para ajudar e alinhar todos os desenvolvedores. Com o \textit{style guide} é possível manter a consistência visual dentro do projeto. Para a construção do \textit{style guide} do \textit{e-commerce} o foco se deu em três informações básicas:


{\textbf{Cores}}

Quando o foco é sobre as cores é preciso ter em mente contraste e luminosidade. Existe algumas deficiências e limitações visuais que fazem com que a cor sofra mudanças então é preciso que o \textit{e-commerce} tenha uma paleta de cores acessível que garanta o contraste ideal entre as cores \ref{Diretrizes nível AA}. No exemplo abaixo da figura \ref{fig1:style} está sendo demonstrado uma paleta de cores inclusiva criada pela ferramenta \textit{Adobe Color Wheel} \cite{ADOBE} . 
\begin{figure}[ht]
  		\centering
  		
        \includegraphics[width=0.8\textwidth]{images/paleta_de_cores_acessiveis.png}
        \caption{Paleta de cores acessíveis construída pelo site da \textit{Adobe Color} \cite{ADOBE}.}
        \caption*{Fonte: a autora (2022)}
        \label{fig1:style}
\end{figure}  

É de extrema importância verificar como ficariam essas cores pela visão de pessoas que possuem algum tipo de daltonismo. Ao perceber que a paleta de cores ainda está com um contraste aceitável entre o primeiro e o segundo plano de no mínimo 7:1 ela está pronta para o uso, seguindo a diretriz nível AAA de contraste \ref{Diretrizes nível AAA}. Na figura \ref{fig2:style} é possível observar os diferentes tipos de daltonismo e como a paleta de cores se comporta diante das limitações de visão de pessoas com daltonismo. Essa paleta foi criada utilizando as ferramentas de acessibilidade do \textit{Adobe Color Wheel} \cite{ADOBE}.
 \begin{figure}[ht]
        \centering
    	\includegraphics[width=0.8\textwidth]{images/paleta_daltonismo.png}
        \caption{Comportamento de uma paleta de cores acessível diante dos tipos de daltonismo.}
        \caption*{Fonte: a autora (2022)}
        \label{fig2:style}
\end{figure}  


\newpage

{\textbf{Tipografia}}

A tipografia não é apenas escolher uma fonte legal para o seu projeto, junto com ela devemos levar em consideração vários aspectos que são importantes e foram listados nas diretrizes: 
\begin{itemize}
    \item Espaçamento
    \begin{itemize}
        \item O espaçamento deve ser bem planejado para que seja possível aplicar um zoom de 200\%\ ou ainda em caso de redimensionamento de tela o texto deve permanecer legível. \ref{Diretrizes nível AA}
    \end{itemize}
    \item Altura de linhas
    \begin{itemize}
        \item Sempre que houver um redimensionamento os textos não devem perder legibilidade. \ref{Diretrizes nível AA}
    \end{itemize}
    \item Hierarquia tipográfica
    \begin{itemize}
        \item Sempre que o conteúdo da tela for dividido em sessões, todas devem possuir títulos claros, com níveis de hierarquia bem definidos, para assim manter a hierarquia da informação de uma forma concisa. \ref{Diretrizes nível AAA}
    \end{itemize}
    \item Pesos, cores e contraste. 
    \begin{itemize}
        \item O contraste entre textos e telas também deve ser mantido num padrão de no mínimo 4:5:1 entre o primeiro e o segundo plano da tela. \ref{Diretrizes nível A}
    \end{itemize}
\end{itemize}

A figura \ref{fig3:tipografia} foi criada usando os recursos de acessibilidade presentes na ferramenta \textit{Adobe Color Wheel} \cite{ADOBE}. A partir da cor do texto e da cor de fundo é possível saber a proporção de contraste e se está dentro do esperado nas diretrizes WCAG \cite{WCAG20}.

\begin{figure}[ht]
    \centering
	\includegraphics[width=0.8\textwidth]{images/contrast_check.png}
    \caption{Checagem de contraste entre texto e fundo realizado por meio do site  \textit{Adobe Color Wheel} \cite{ADOBE}}
    \caption*{Fonte: a autora (2022)}
    \label{fig3:tipografia}
\end{figure} 

\newpage
{\textbf{Elementos da \textit{interface} do usuário}}

Este tópico compreende os elementos que estão presentes na \textit{interface}, logo é apresentados exemplos e listagem desses componentes. Na construção desses elementos é feito o detalhamento visual, ou seja, os tamanhos, espaços, margens, cores que ele deve possuir como também os seus estados comportamentais, se houver um erro como ele se comporta, em caso de elemento focado como deve ficar o elemento.

Com o componente \textit{Card} da figura \ref{fig:UI_componentes} abaixo é possível analisar onde as diretrizes foram aplicadas e como elas impactam na construção de componentes mais complexos da \textit{interface} do usuário.  

\begin{figure}[!htb]
    \centering
	\includegraphics[width=0.38\textwidth]{images/ui-exemplo.png}
    \caption{Exemplo de um componente montado com elementos da interface do usuário}
    \caption*{Fonte: a autora (2022)}
    \label{fig:UI_componentes}
\end{figure}

No quadro abaixo \ref{elementos de UI} foi referenciado os identificadores de 1 a 8 apontados na Figura \ref{fig:UI_componentes} com as diretrizes aplicadas em cada ponto juntamente com sua breve descrição. Foi especificado o nível em que cada diretriz se encontra. Para maiores detalhes sobre as diretrizes consultar o apêndice \ref{sec:apendice}.
{


\noindent\begin{minipage}{\linewidth}
\centering
\captionof{table}{Lista de identificadores de 1 a 8 e as diretrizes contempladas}
% \caption{Lista de identificadores de 1 a 8 e as diretrizes contempladas}
\resizebox{\linewidth}{!}{%
\begin{tabular}{|c|l|l|c|p{400px}|} 
\hline
\rowcolor[HTML]{ECF4FF} 
{\color[HTML]{333333} Identificador & Diretriz} & Nome & Nível & Descrição \\ 
\hline
    1 & 1.3.1 & Informações e Relações & A & As estruturas da tela devem ser construída de forma que sua arquitetura de informação faça sentido tanto para todos, sejam ouvintes ou leitores.\\ 
        \cline{2-5} & 1.3.2 & Sequência com significado & A & A apresentação das informações na tela sempre deverá ter uma sequência lógica. \\ 
        \cline{2-5} & 2.4.7 & Foco visível & A & Ao se interagir por teclado, qualquer pessoa deve conseguir identificar qual é a sua localização na tela através de um foco visível. \\ 
        \cline{2-5} & 3.2.1  & Em foco & A & O foco sempre deve se manter durante a navegação, guiando o usuário e evitando mudanças que causem desorientação.\\
       \hline
    2 & 1.1.1 & 
        Conteúdo Não Textual & A & Todo o conteúdo não textual que é exibido ao usuário tem uma alternativa textual que serve a um propósito equivalente.\\
        \cline{2-5} & 1.4.11 & Contraste Não textual & AA & apresentação visual a seguir tem um relação de contraste de pelo menos 3:1 contra cor(es) adjacente(s).\\
        \hline
    3 & 1.4.3 & 
        Contraste (mínimo) & AA & Textos devem ter uma relação de contraste entre primeiro e segundo plano de ao menos 4.5:1.\\
        \cline{2-5} & 2.4.6 & Cabeçalhos e rótulos  & AA & Todos os títulos e rótulo devem descrever claramente a finalidade dos conteúdos, não deve haver ambiguidade em seu entendimento.\\
    \hline
    4 & 1.4.6 & 
        Contraste (melhorado) & AAA & Textos devem ter uma relação de contraste entre primeiro e segundo plano de ao menos 7:1.\\
        \cline{2-5} & 1.4.11 & Contraste Não textual & AA & A apresentação visual a seguir tem um relação de contraste de pelo menos 3:1 contra cor(es) adjacente(s).\\
    \hline 
     5 & 1.4.12 & Espaçamento de texto & AA & Sempre que houver um redimensionamento os textos não devem perder legibilidade\\

        \cline{2-5} & 1.4.13 & Conteúdo em foco por mouse ou teclado & AA & Conteúdos adicionais não devem ser acionados apenas com foco por mouse ou teclado.\\
        \cline{2-5} & 2.1.1 & Teclado & A & Todas as funcionalidades devem ser acionadas via teclado. \\
        \cline{2-5} & 2.1.2 & Sem bloqueio de teclado & A & Ao se interagir via teclado, a navegação por todos os elementos ”clicáveis”deve ocorrer sem que haja bloqueios ou interrupções.\\
        \cline{2-5} & 2.1.3 & Teclado (sem exceção) & AAA & Todas as funcionalidades devem ser acionadas via teclado, sem exceção.\\
        \cline{2-5} & 2.4.7 & 
        Foco visível  & A & Ao se interagir por teclado, qualquer pessoa deve conseguir identificar qual é a sua localização espacial na tela através de um foco visível.\\
        \cline{2-5} & 2.5.3 & Rótulo no Nome acessível & A & Rótulos em botões, ícones acionáveis ou qualquer controle interativo, devem ter uma descrição significativa.\\
         \cline{2-5} & 2.5.5 & Tamanho da área clicável & AAA & O tamanho das áreas acionáveis por clique ou toque devem possuir no mínimo 44x44 pixeis de espaçamento\\
        \cline{2-5} & 3.2.1 & Em foco & A & O foco sempre deve se manter durante a navegação, guiando o usuário e evitando mudanças que causem desorientação.\\

    \hline 
    6 & 1.4.12 & Espaçamento de texto & AA & Sempre que houver um redimensionamento os textos não devem perder legibilidade\\
        \cline{2-5} & 1.4.13 & Conteúdo em foco por mouse ou teclado & AA & Conteúdos adicionais não devem ser acionados apenas com foco por mouse ou teclado.\\
        \cline{2-5} & 2.1.1 & Teclado & A & Todas as funcionalidades devem ser acionadas via teclado. \\
        \cline{2-5} & 2.1.2 & Sem bloqueio de teclado & A & Ao se interagir via teclado, a navegação por todos os elementos ”clicáveis”deve ocorrer sem que haja bloqueios ou interrupções.\\
        \cline{2-5} & 2.1.3 & Teclado (sem exceção) & AAA & Todas as funcionalidades devem ser acionadas via teclado, sem exceção.\\
        \cline{2-5} & 2.4.3 & Ordem do foco & A & A interação por elementos focáveis na tela sempre deverá ser sequencial e lógica de acordo com o conteúdo apresentado.\\
        \cline{2-5} & 2.4.7 & Foco visível & A & Ao se interagir por teclado, qualquer pessoa deve conseguir identificar qual é a sua localização na tela através de um foco visível. \\
        \cline{2-5} & 2.4.8 & Localização & AAA & Qualquer pessoa deve conseguir se localizar ou se orientar facilmente em qualquer nas telas.\\
        \cline{2-5} & 2.5.2 & Cancelamento de acionamento & A & Deve se fornecer um modo de cancelar acionamentos feitos de forma não proposital.\\
        \cline{2-5} & 3.2.3 & Navegação consistente & AA & Deve-se manter a consistência com relação ao formato de apresentação, interação e localização na tela. \\
        \cline{2-5} & 3.2.4 & Identificação consistente & AA & Deve-se manter a consistência com relação a diferentes formatos de elementos, mas que possuem uma mesma funcionalidade.\\
    \hline
    7 & 1.4.3 & 
        Contraste (mínimo) & AA & Textos devem ter uma relação de contraste entre primeiro e segundo plano de ao menos 4.5:1.\\
        \cline{2-5} & 1.4.6 & Contraste (melhorado) & AAA & Textos devem ter uma relação de contraste entre primeiro e segundo plano de ao menos 7:1.\\
    \hline
    8 & 4.1.3 & 
        Mensagens de status & AA & Qualquer tipo de mensagem que é resultado de uma ação deve ser transmitida sem que ocorra uma mudança de foco na tela.\\
    \hline

\end{tabular}}
\label{elementos de UI}

\end{minipage}

}}


\subsection{Fase 2: Adaptação e construção da \textit{interface} do usuário acessível e usável com base no guia de acessibilidade e \textit{style guide}}
{A segunda fase se concentrou na adaptação da \textit{interface} do usuário e na construção de novos componentes para o \textit{e-commerce}, sempre seguindo os guias, como exemplificado na figura \ref{guideline-f3}. Nessa fase o foco principal se dividiu em duas vertentes:
\begin{itemize}
\item Criar abordagens inteligentes baseadas em casos de uso para problemas de acessibilidade em imagens, formulários e navegação e a adaptação de componentes com \textit{HyperText Markup Language} (HTML) \cite{flatschart2011html}  não semântico para semântico.
\item  Adaptar e criar componentes para que as tecnologias assistivas pudessem entendê-los corretamente e assim ter uma a navegação por teclado eficaz.

\end{itemize}

\begin{figure}[ht]
    \centering
	\includegraphics[width=0.8\textwidth]{images/guideline-f2.png}
    \caption{Fase 2 do guideline}
    \caption*{Fonte: a autora (2022)}
    \label{guideline-f3}
\end{figure} 
}
\subsubsection{Abordagens inteligentes e adaptação de componentes}
{

O primeiro passo a se fazer foi organizar o \textit{HTML} de forma lógica e semântica, onde cada \textit{tag} deve ser usada para o fim que ela foi criada, seguindo uma ordem que seja compreensível além de corresponder ao conteúdo desejado. Nesse ponto em um sistema legado muitos componentes estavam escritos de maneira errada, algumas \textit{tags} \textit{HTML} estavam sendo usadas com um proposito diferente do que ela foi criada e para esses casos foi utilizado a abordagem \textit{Accessible Rich Internet Applications Suite} (WAI-ARIA) \cite{WAI-ARIA} para que o leitor de tela conseguisse entender o que era cada \textit{tag} daquela e também fosse possível fazer a navegação via teclado.

{\textbf{Caso de uso 1}: Transformando botão inacessível em acessível}

Adaptação de um botão que foi construído de forma não acessível, como exemplificado no código \ref{code1:bad},  e por isso não era lido por leitores de tela e nem navegável via teclado. No exemplo do código \ref{code2:good} foi usado alternativas para deixar o botão acessível, foi adicionado o atributo \lstinline{role} do \textit{WAI-ARIA}\cite{WAI-ARIA} para dar o comportamento de botão a essa \lstinline{div} e com isso ganhar todos os atributos e comportamentos que um botão teria.

\begin{lstlisting}[language=html,caption={Componente de botão antes de receber boas práticas de acessibilidade.},  label=code1:bad]
    <div
        className="button__add"
        onClick={event => onClick(event)}
    >
      Adicionar ao carrinho 
    </div>

\end{lstlisting}}
{\begin{lstlisting}[language=html,caption={Adaptação do componente de botão usando o atributo \textit{role} presente no \textit{WAI-ARIA} \cite{WAI-ARIA}.}, label=code2:good]
<div 
    role="button" 
    tabIndex="0" 
    className="button__add"
    onClick={event => onClick(event)}
    onKeyDown={event => keyHandler(event, props.onClick)}
    {...props}
>
    Adicionar ao carrinho
</div>
 
\end{lstlisting}}

\vspace{1.5cm}
{\textbf{Caso de uso 2}: Adaptação e construção de imagens acessíveis.
As imagens que possuem algum tipo de conteúdo devem ser acessíveis, elas precisam de alguma descrição, seja ela visível ou não. Existe o atributo \lstinline{alt} para a \textit{tag} \lstinline{img}, como exemplificado no código \ref{code3:good}, onde ele serve para descrever o que está presente na imagem. Essa descrição do \lstinline{alt} não aparece visualmente, mas ela é lida pelo leitor de tela, quando o usuário, navegando pelo teclado, passar pela imagem.}
{\begin{lstlisting}[language=html,caption={Uso atributo \lstinline{alt} para descrever o item da imagem.}, label=code3:good]
<img src="headphone.jpg" alt="Headphone na cor preto"> 
\end{lstlisting}}
Caso haja necessidade de fornecer uma informação contextualizada é necessário fazer uso do atributo \lstinline{title} e nele colocar a informação desejada. Nesse caso, a maioria dos leitores de tela lerá o texto alternativo, o atributo de título e o nome do arquivo. Além disso, os navegadores exibem o texto do título como dicas de ferramentas quando estão sobre o \textit{mouse}, ou seja quando o mouse está parado em cima da \textit{tag que possui esse atributo} . No exemplo do código \ref{code4:good} foi aplicado duas \textit{tags} semânticas presentes no \textit{HTML 5}, esses dois elementos o \lstinline{<figure>} e \lstinline{figcaption>}, servem para associar uma figura a uma legenda. 
{\begin{lstlisting}[language=html,caption={Uso do atributo \textit{alt} em conjunto com as \textit{tags} \textit{figure} e \textit{figurecaption}.}, label=code4:good]
<figure>
  <img src="headphone.jpg" alt="Headphone na cor preto"> 
  <figcaption>Headphone com microfone acoplado na cor preta, com um fio de ótima construção</figcaption>
</figure>
\end{lstlisting}}

Ainda há os casos onde as imagens são apenas decorativas e por elas não carregarem conteúdo elas devem ser ignoradas pelos recursos de tecnologia assistivas. Para esse caso há três tipos de abordagens: 
\begin{itemize}
\item Por ser decorativa a melhor abordagem é incluí-las na página como imagens de fundo através de \textit{Cascading Style Sheets} (CSS) \cite{CSS}, como exemplificado no código \ref{code5:good} e caso a imagem transmita significado ao conteúdo mas tenha sido inserida via \textit{CSS}, pode-se utilizar recursos \textit{WAI-ARIA}, exemplificado no código \ref{code7:good} .
{\begin{lstlisting}[language=html,caption={Adição de imagens via CSS.}, label=code5:good]
<style>
div {
    background-image:url("./images/logotipo.png");
}
</style>
\end{lstlisting}}
\item Em casos de imagens que usam a \textit{tag} \lstinline{img} basta ter uma descrição \lstinline{alt} vazia. Isso faz com que os leitores de tela reconheçam a imagem mas não tentem descreve-la, como exemplificado no código \ref{code6:good} abaixo. A razão para usar um \lstinline{alt} vazio ao invés de não incluí-lo é porque muitos leitores de tela anunciam o \textit{Uniform Resource Locator} (URL) da imagem inteira se nenhum \lstinline{alt} for fornecido.
{\begin{lstlisting}[language=html,caption={Uso do atributo \lstinline{alt} vazio.}, label=code6:good]
<img src="logotipo.jpg" alt=""> 
\end{lstlisting}}

\item Como exemplificado no código \ref{code7:good} em casos de imagens incluídas em \textit{tags} que não são \lstinline{img} aconselha-se fazer uso do atributo \textit{WAI-ARIA role} \lstinline{(role="presentation")}, isso também impede que os leitores de telas leiam textos alternativos. 
{\begin{lstlisting}[language=html,caption={Uso atributo WAI-ARIA \cite{WAI-ARIA} role presentation.}, label=code7:good]
<div role="presentation" src="logotipo.jpg"/>
\end{lstlisting}}

\end{itemize}

{

{\textbf{Caso de uso 3:} Criando formulários acessíveis.} 

O primeiro passo para criar um formulário acessível é criar uma estrutura \textit{HTML} com sequência lógica, essa sequencia é definida pela ordem que o \textit{HTML} é escrito. Fazendo isso a navegação pelos campos do formulário ficará mais fácil tanto por teclado quanto para leitores de tela. Para se construir formulários acessíveis deve-se tomar os seguintes cuidados na hora da implementação: 
\begin{itemize}
    \item A associação de um campo de entrada a um rótulo fará diferença para o usuário final. E quando não for possível fazer essa associação por tags \textit{HTML} existe outras opções como a forma \textit{WAI-ARIA}  usando o atributo \lstinline{aria-label}.
    \item É necessário o fornecimento de instruções claras sobre os dados de entradas pedidos no campo. As entradas devem ser facilitadas, os caracteres especiais em campos numéricos devem ser retirados etc. 
    \begin{itemize}
        \item Para campos com obrigatoriedade de dados, deve-se ter um indicador visual e escrito de obrigatoriedade. O atributo \lstinline{required} auxilia no quesito de obrigatoriedade de um campo.
        \item Caso necessário há a possibilidade de se adicionar o atributo de estado\lstinline{aria-invalid} para indicar se o campo está válido ou inválido e ele pdoe ter os valores \textit{true} ou \textit{false} .
    \end{itemize}
    \item A identificação e descrição de erros de entrada e confirmação de envio de informações devem ser claras para o usuário. 
    \begin{itemize}
        \item A indicação do erro é feita de três formas: com uma cor, um texto informativo e um ícone de alerta. Assim, não é usada somente uma forma para representar esta informação, beneficiando também pessoas com limitações visuais e com deficiências cognitivas que usam leitor de telas ou que possuem daltonismo;
        \item Uma área para mensagem informativa em um formulário sempre será bem-vinda. Esta área de mensagem informativa do formulário possui os atributos \lstinline{role="alert"}, que avisa que é uma mensagem de alerta e \lstinline{aria-live="assertive"} que informa que é um campo que pode sofrer modificações no conteúdo e o leitor de telas é informado quando há modificação neste campo;
    \end{itemize}
    \item É necessário haver distinção visual de qual campo está recebendo o foco no momento, mesmo aplicando o \lstinline{outline} padrão do navegador. A customização desse \lstinline{outline} pode ser feita para que ele sempre fique visível qual campo contém o foco.
\end{itemize}

 O código construído no exemplo \ref{code8:good} é possível observar todas as enumerações da lista acima desde a construção do HTML \cite{HTML} ordenado, passando por associação de rótulos a campos e instruções sobre o tipo de entrada que é esperado como também o \textit{feedback} conciso para eventuais indicações de erros. Como resultado do código \ref{code8:good} após a compilação do código desenvolvido temos a imagem \ref{fig5:code8} que apresenta de forma visual todas as enumerações desenvolvidas no código.
    
{\begin{lstlisting}[language=html, caption={Formulário acessível usando abordagens citadas acima}, label=code8:good]
<form>
    <output id="formMessage" role="alert" aria-live="assertive" tabindex="0" class="error">
        O formulário apresenta erros que impedem a finalização do seu cadastro. Confira se todos os campos obrigatórios foram preenchidos e tente novamente.
    </output>

    <p>
        <label for="email1">
            E-mail <span class="required">(obrigatório)</span>:
        </label>
        <input type="email1" name="email1" id="email1" required="" aria-invalid="true" aria-describedby="emailMessage">
        <span class="fieldMessage" id="emailMessage">
            O e-mail não pode ficar em branco e deve ser válido.
        </span>
      </p>

      <p>
        <label for="phone1">
            Telefone <span class="optional">(opcional)</span>:
        </label>
        <input type="tel" name="phone1" id="phone1" aria-describedby="phoneTip">
        <span class="fieldTip" id="phoneTip">
            Formato: (99) 99999-9999.
        </span>
      </p>

    <button class="formButton" id="formButtonARIA" type="submit">Finalizar cadastro</button>
</form>
\end{lstlisting}}

\begin{figure}[ht]
  	\center
    \includegraphics[width=0.7\textwidth]{images/form-acessivel.png}
    \caption{Exemplifica o código \ref{code8:good} após a compilação.}
    \caption*{Fonte: a autora 2022}
    \label{fig5:code8}
\end{figure} 
}
\subsubsection{Adaptação de componentes para o uso de leitores de telas e navegação por teclado}
{Antes de falar de adaptação de componentes é necessário entender como funciona o leitor de telas. O leitor de telas basicamente é um software que vai percorrendo textos e imagens e lendo em voz alta tudo o que ele encontra na tela, a partir do momento que o usuário começa a interagir com o software para efetuar uma navegação ele vai percorrendo o documento saltando entre elementos interativos e cabeçalhos. Para que essa navegação ocorra tudo deve está funcionando direito, pois caso contrário o usuário pode se perder na página e ter dificuldades em entender como a informação está organizada. No leitor de telas a navegação ocorre de três maneiras: 
\begin{itemize}
    \item Navegação com as setas direcionais: Usando a navegação por setas direcionais o usuário consegue acessar as informações textuais.
    \item Navegação por tecla \textit{tab}: O leitor sai pulando para os links espalhados na página.
    \item Navegação com a tecla h: O leitor salta entre os cabeçalhos marcados na página.
\end{itemize}

{

{\textbf{Caso de uso 1:} Navegação por cabeçalho.}

Cabeçalhos compõem a hierarquia de informação do site, essa hierarquia é usada pelos leitores de tela para navegação dentro da página. A grande maioria dos leitores de tela dispõem de atalhos para acesso rápido aos cabeçalhos marcados na página. Para garantir que tecnologias assistivas acessem o conteúdo dos cabeçalho sem sua ordem de importância é preciso tomar alguns cuidados: 
\begin{itemize}
    \item Cuidado com múltiplas \textit{tags} \lstinline{h1} na página. Mesmo que haja separação de contexto para a utilização de mais de um \lstinline{h1} por página o leitor de tela não consegue identificar a diferença e a importância de \lstinline{h1} contidos em outras \textit{tags}.
    \item Para essa navegação funcionar, o código precisa ter sido escrito com a semântica correta, ou seja, cabeçalhos devem ser definidos como cabeçalhos e não usar qualquer outro artifício visual para se assemelhar a um cabeçalho, como exemplificado abaixo. 
\end{itemize}
{\begin{lstlisting}[language=html,caption={cabeçalhos com hierarquia}, label=code9:good]
<h1>Cabeçalho Principal</h1>
<p>Exemplo de um texto de um parágrafo</p>
<h2>Cabeçalho Secundário</h2>
<p>Mais texto de um parágrafo.</p>
\end{lstlisting}}


{\textbf{Caso de uso 2:}  Navegação por tabulação e leitores de tela}

A ordem da navegação por tabulação é derivada da forma como o \textit{layout} do \textit{site} está escrito, contudo a ordem padrão pode não corresponder necessariamente a ordem visual. Ao utilizar a tecla \textit{tab}, o leitor de tela realiza o foco e lê em voz alta os \textit{links} e componentes interativos presentes na página.  A tabulação é utilizada somente para navegar entre os \textit{links}, e com isso todos os \textit{links} terão que ser localizados através destas duas formas, com a tabulação ou com as setas. O que diferencia uma forma da outra, é que com as setas, o usuário poderá ler toda a página (incluindo texto entre os \textit{links}), e com a tabulação, o usuário só pode localizar os \textit{links}, botões, campos de edição, e caixas de seleção.

Navegar por entre os \textit{links} é uma forma de observar o texto rapidamente, especialmente se os usuários estiverem tentando encontrar uma seção específica do \textit{website}. Para ter uma navegação por tabulação eficiente é preciso passar por alguns pontos: 

\begin{enumerate}
    \item Ordene de forma lógica e intuitiva a leitura e tabulação. Preparar o \textit{HTML}  para ter uma navegação consistente e eficaz é essencial para o bom funcionamento da navegação atrelada a leitores de tela. 
    \item Usar a tecla \textit{tab} para passar pelos \textit{links} e controles de formulários
    das páginas, certificando-se de que todos os links e controles de formulários podem ser acessados, bem como se os links indicam claramente para onde levam. 
    
    \item Com atributo \lstinline{tabindex} é possível definir uma ordem de foco nos elementos dispostos na tela. Usando o \lstinline{tabindex=0} é inserido um elemento na ordem natural de tabulação, como visto no exemplo do código \ref{code10}, e com \lstinline{tabindex="-1"} um elemento é removido da ordem e tabulação.
    {\begin{lstlisting}[language=html,caption={Uso do atributo \textit{tabindex} para adicionar um elemento a ordem de foco}, label=code10]
        <label>
            Primeiro na lista de tabulação:<input type="text">
        </label>
        <div tabindex="0">
        Próximo item na lista de tabulação, mesmo não sendo um elemento que receberia o foco natural
        </div>
        <div>Não será focado pois está sem o tabindex</div>
    \end{lstlisting}}
    
    \item Adicione o atributo \lstinline{title} aos \textit{links}, ele mostrará uma descrição do lugar pra onde o \textit{link} leva, melhorando a navegação. Ao ter o elemento focado o leitor conseguirá ler o atributo \lstinline{title} e saber a direção que o redirecionamento poderá tomar ou se um elemento novo surgirá, como exemplificado no código \ref{code11}.
    {\begin{lstlisting}[language=html,caption={Uso do atributo \textit{title} para dar uma explicação sobre onde o link leva}, label=code11]
    <a href="#" onclick="abrePopup()" onkeypress="abrePopup()" title="Abre uma janela pop-up com Javascript">Ver mais informações</a>
    \end{lstlisting}}
    
    \item No exemplo exposto no código \ref{code11} é usada a abordagem de \textit{landmarks}, que consiste no  fornecimento de âncoras para ir direto a um bloco de conteúdo. São comumente chamados de \textit{landmarks} é um tipo de região em uma página \textit{web} a qual uma pessoa pode querer acessá-la rapidamente. Um dos principais benefícios em se usar pontos de referência é a possibilidade de o usuário acessar diretamente uma região da página sem a necessidade de seguir a ordem natural do conteúdo. Isso é especialmente relevante para o caso de pessoas que utilizam leitores de tela para navegar na \textit{web}. Para essa abordagem é muito utilizado o CSS \cite{CSS} deixando-os \textit{links} invisíveis e apenas disponíveis quando a navegação por tabulação se iniciar.
    
    {\begin{lstlisting}[language=html,caption={Criação de \textit{landmarks} para acessar pontos de referências através do atalho p.}, label=code12]
    <div role="heading" id="cabecalho"> 
      <h1>O Cabeçalho</h1> 
      <a href="#conteudo" accesskey="p">Pular para o conteúdo principals</a> 
    </div> 
    <div role="navigation" id="navegacao"> conteúdo da navegação </div>
    <div role="main" id="conteudo"> conteúdo da main </mdiv>
    \end{lstlisting}}
    
    \item É aconselhável não tabelas para diagramação. As tabelas devem ser utilizadas apenas para dados tabulares e não para efeitos de disposição dos elementos na página. O leitor de telas sairá pulando por cada campo das tabelas e o usuário poderá ficar perdido na navegação. Para efeitos de diagramação a melhor opção está na utilização do \textit{CSS}.
    
    \item Utilização de separação dos \textit{links} adjacentes. Em uma sequência de \textit{links}, além do espaço, é importante o uso de separadores ou elementos do HTML adequados para que as pessoas com deficiência identifiquem claramente onde termina e começa um novo \textit{link}. É recomendado o uso de listas, assim como no exemplo do código \ref{code13} , em que cada elemento dentro da lista é um link.
    {\begin{lstlisting}[language=html,caption={Separação de \textit{links} adjacentes usando a \textit{tag} li.}, label=code13]
    <ul id="menu">
        <li> <a href="home.html">Home</a></li>
        <li> <a href="pesquisa.html">Pesquisa</a></li>
    </ul>
    \end{lstlisting}}
    
    \item É desencorajado a abertura novas instâncias sem a solicitação do usuário. O impedimento de  redirecionamento automático de páginas é algo necessário. Forçar a abertura de \textit{links} em uma nova janela não é algo esperado e o usuário se perderá na navegação. Caso implemente essa funcionalidade, o \textit{link} deve possuir uma iconografia óbvia e explicativa, usando pseudo-elemento \lstinline{::after} no \textit{CSS}. No \textit{CSS}  o conteúdo de pseudo-elementos é lido em voz \textit{alt} a por leitores de tela, portanto é de extrema importância de que o conteúdo seja com informações relevantes.
    
    \item O conteúdo deve possuir o foco por mouse ou teclado e esse foco deve ser visível. Para pessoas com baixa visão, é muito importante que seja possível perceber facilmente onde está o foco do teclado, garantindo uma maior facilidade de navegação. Como no exemplo do código \ref{code14} pseudo-classe \lstinline{:focus} é utilizada para definir o estilo de qualquer elemento HTML que receber o foco do teclado, como \textit{links} e elementos de formulário.
    {\begin{lstlisting}[language=html,caption={Exemplo de como estilizar o o elemento nos estados de foco e hover.}, label=code14]
    a:focus, a:hover {
        border: 2px solid #F00;
    }
    \end{lstlisting}}
    
    \item Todas as funcionalidades que estão disponíveis via mouse deverão estar disponíveis para serem acionadas via teclado. Só dessa forma o site estará acessível para pessoas com deficiências visuais.
    
\end{enumerate}

}}



\subsection{Fase 3: Construção do dashboard acessível}
{

Para a fase de construção do dashboard acessível o foco se manteve na construção dos gráficos de forma acessível, navegável e reutilizável como exemplificado na figura \ref{guideline-f3}.
\newpage

\begin{figure}[ht]
  	\center
    \includegraphics[width=0.7\textwidth]{images/guideline-f3.png}
    \caption{Fase 3 do guideline}
    \caption*{Fonte: a autora 2022}
    \label{guideline-f3}
\end{figure} 

Para a construção dos gráficos do \textit{dashboad} foi utilizada a biblioteca D3.js \cite{D3} que é usada para visualização de informação. Essa biblioteca permite que sejam construídas aplicações em que os dados entram puros e são dinamicamente associados em representações gráficas como exemplificado em \ref{graph:d3}. Com o D3 o \textit{Document Object Model} (DOM) \cite{DOM} pode ser facilmente manipulado já que ele faz uso dos padrões da \textit{web} como \textit{HTML}, \textit{CSS} e \textit{Scalable Vector Graphics} (SVG) para renderizar gráficos de visualização poderosos.

\begin{figure}[ht]
  	\center
    \includegraphics[width=0.6\textwidth]{images/exemplo-graf-d3.png}
    \caption{Um gráfico desenvolvido com D3.js}
    \caption*{Fonte: a autora 2022}
    \label{graph:d3}
\end{figure} 

}

% \vspace{1cm}
{\textbf{Construção de gráficos com D3.js}}



No \textit{e-commerce} foi decidido não construir o gráfico inteiramente com D3.js, pois a abordagem normal presente não facilitaria o reaproveitamento para a criação dos componentes. Quando um gráfico é plotado somente com D3.js  pode haver problemas de performance, visto que o React \cite{REACT} não estava preparado para esse tipo de renderização contínua criada ao se aplicar filtros e interações nos gráficos. Dessa forma, a abordagem mais coerente era usar as \textit{tags} \textit{HTML} normais e quando era necessário feito a criação do elementos e interações usando o D3.js.

Para a manipulação do DOM, é necessário aprender a utilizar os recursos presente no D3. A biblioteca usa JavaScript para realizar a maioria das tarefas de seleção, transição e vinculação de dados.

A primeira coisa necessária para construir um gráfico com D3.js é adicionar a biblioteca ao projeto, seja por gerenciador de pacotes ou por importação direta na \textit{tag} \lstinline{script}. Feito isso, é preciso ter uma \textit{tag} como base onde as partes do gráfico serão injetadas, geralmente é usada a \textit{tag} \lstinline{svg} pois ela auxilia na construção de formas geométricas que será utilizado no gráfico. Após adicionar a \textit{tag} \lstinline{svg} precisamos adicionar uma nova \textit{tag} \lstinline{g} para que o gráfico comece a tomar forma, essa \textit{tag} \lstinline{g} é um recipiente utilizado para agrupar objetos.

Para começar a desenhar qualquer tipo de gráfico é necessário colocar os eixos das abscissas e ordenadas. Na construção desses eixos deve-se estabelecer qual o domínio de cada eixo e após isso iniciar a construção das escalas, para que assim todos os dados se comportem dentro dos limites estabelecidos em função da escala. O próprio D3  já fornece alguns tipos de funções que auxiliam na criação dessas escalas, as funções são: \lstinline{d3.scaleBand()}, esta função divide o intervalo em n bandas onde n é o número de valores na matriz de domínio e \lstinline{d3.scaleLinear}, que é usada para criar um ponto de escala visual. São criado duas escalas, uma para o eixo X usando a \lstinline{d3.scaleBand()} e outra para o eixo Y usando a \lstinline{d3.scaleLinear}.

O próximo passo é criar por fim o conteúdo do gráfico de forma visual. Com o apoio de dados vindos de uma API é possível atribuir ao longo do eixo X e do eixo Y os valores e formas visuais necessárias. Utilizando a função \lstinline{attr} é definido os atributos do elemento, uma vez que essa função recebe dois parâmetros o nome do atributo e o valor desse atributo. Graças a essa função são definidas a altura, largura e os valores de cada ponto dos eixos.

Depois que todos os valores dos gráficos estiverem corretamente plotados é o momento de lidar com as interações que estarão no gráfico. Todas as interações presentes via mouse devem também estarem disponíveis via teclado, ou seja, se houver um interação de clique de mouse deve haver uma interação de pressão de tecla. Para auxiliar na construção dessas interações pode-se usar a função \lstinline{.on()} que ficará responsável por agir como um \textit{event listener} de um elemento, onde qualquer tipo de interação será percebida. Essa função recebe dois argumentos, sendo o primeiro o tipo de evento e o segundo uma função que será disparava quando essa evento for detectado. 

{\textbf{Tornando os gráficos acessíveis a leitores de tela}}

{Além de tornar os gráficos acessíveis a qualquer tipo de navegação e interação é preciso que as imagens e os elementos presentes também estejam acessíveis e para isso existem algumas abordagens eficientes.  

O gráfico exemplificado abaixo \ref{fig:grafico-d3} foi construído levando em consideração as barreiras de acessibilidade que um deficiente visual frequentemente se depara. Logo, para diminuir essas barreiras as decisões tiveram como apoio as diretrizes WCAG que se adéquam ao propósito do \textit{e-commerce} e a construção do gráfico com D3.js se deu com o apoio das mesmas. 


\begin{figure}[ht]
  	\center
    \includegraphics[width=0.7\textwidth]{images/barchart-tooltip.png}
    \caption{Gráfico do tipo de barras estacadas.}
    \caption*{Fonte: a autora 2022}
    \label{fig:grafico-d3}
\end{figure} 

}


{

{Tomando como exemplo a figura \ref{fig:grafico-d3}, a tabela \ref{Barreiras e decisões na construção de um gráfico acessível} logo abaixo exemplifica cada barreira de acessibilidade e sua respectiva tomada de decisão.}


\noindent\begin{minipage}{\linewidth}
\centering
\resizebox{\linewidth}{!}{%
\begin{tabular}{|l|l|p{400px}|} 
\hline
\rowcolor[HTML]{ECF4FF} 
{\color[HTML]{333333}  Barreira de acessibilidade} &
  Tomada de decisão  \\ 
\hline
\begin{tabular}[c]{@{}l@{}}Falta de atributo \lstinline{alt} ou atributo não \\ descritivo\end{tabular} &
  \begin{tabular}[c]{@{}l@{}}Se faz necessário o uso do atributo \lstinline{alt}. Os leitores de tela falam texto alternativo sem permitir que os usuários acelerem \\ou pulem,  portanto, as informações sejam descritivas, mas sucintas.\end{tabular} \\ \hline
Falta de rotulação nos dados &
  \begin{tabular}[c]{@{}l@{}}Legendas podem ser não eficientes para usuários daltônicos ou deficientes visuais que podem ter dificuldade em \\ combinar as cores do gráfico  com as da legenda.  Rotulando também diminuirá o trabalho de varredura para frente \\ e para trás tentando combinar a legenda com os dados. \end{tabular}\\ \hline
Dados fornecidos apenas por gráficos &
  \begin{tabular}[c]{@{}l@{}}Forneça os dados de algum outro modo que não somente por gráficos: Incluir um link para um CSV ou outro formato \\ de dados legível por máquina  para que pessoas com deficiência visual possam navegar pelos dados do gráfico com\\ um leitor de tela é essencial.\end{tabular} \\ \hline
  
\begin{tabular}[c]{@{}l@{}}Pouco contraste entre o fundo do \\ gráfico e o conteúdo do gráfico\end{tabular} &
  \begin{tabular}[c]{@{}l@{}}Verifique se o contraste de cores do gráfico está dentro dos padrões da WCAG .\end{tabular} \\ \hline
  
\begin{tabular}[c]{@{}l@{}}Separação não clara entre os \\setores dos gráficos\end{tabular} &
  \begin{tabular}[c]{@{}l@{}}Utilizar espaços em branco para separar gráficos que são empilhados.  O uso criterioso do espaço em branco aumenta\\ a legibilidade, ajudando a demarcar e distinguir as diferentes seções sem depender apenas da cor. Isso também pode \\ complementar as opções de cores acessíveis,  ajudando os usuários a distinguir a diferença entre as cores que identificam \\seções separadas.\end{tabular} \\ \hline

\end{tabular}%
}
\captionof{table}{Exemplificação de barreiras e decisões na construção de um gráfico acessível}
\label{Barreiras e decisões na construção de um gráfico acessível}

\end{minipage}

}

\vspace*{20px}

{\textbf{Gráficos como componentes reutilizáveis}}


{Quando se trabalha com React ao se criar um componente é preciso ter a garantia de que esse componente possa ser reutilizável, para que isso seja possível no momento da criação do componente deve-se ter em mente que quanto menor e mais genérico for o componente mais ele será escalável e com isso facilitará a construção de componentes maiores.

Para facilitar a construção como módulos separados cada tipo de gráfico se tornou um componente reutilizável. Em cada componente desses contaria com suas funções e particularidades, porém com a flexibilidade de ser um componente genérico, onde todos os dados são populados através de propriedades que são repassadas do componente pai para o componente filho. 



}



\subsection{Fase 4: Testes e avaliações de acessibilidade}
{Durante e após a construção do \textit{e-commerce} de acordo com os
padrões \textit{web} e as diretrizes de acessibilidade, é necessário efetuar testes para garantir sua acessibilidade. Todos os testes descritos nessa seção foram executados por pessoas que não são os \textit{stakeholders} do projeto mas fazem parte da equipe de \textit{quality assurance}. 

Com o objetivo de cobrir o máximo de casos de testes possível as avaliações abordaram dois modos de testes: a avaliação automático e a manual, como exemplificado na figura \ref{guideline-f4}.

\begin{figure}[ht]
  	\center
    \includegraphics[width=0.8\textwidth]{images/guideline-f4.png}
    \caption{Fase 4 do guideline}
    \caption*{Fonte: a autora 2022}
    \label{guideline-f4}
\end{figure} 


\begin{itemize}
\item Avaliação automática: São as avaliações feitas de maneira automática por meio de \textit{plugins} de verificação e de \textit{websites} de validação. Quando os testes automáticos não atingiam os requisitos de aceitação era gerado um relatório contendo os pontos de falhas dentro do teste, esse relatório era repassado para a equipe de desenvolvimento para que as devidas correções fossem feitas.
\begin{itemize}
    \item \href{https://chrome.google.com/webstore/detail/accessibility-developer-t/fpkknkljclfencbdbgkenhalefipecmb?hl=pt-BR}{\textit{Google Accessibility Developer Tools}} ({Plugins} para navegador \textit{Chrome}).
    \item \href{https://developer.chrome.com/docs/devtools/}{\textit{DevTools}} ({Plugins} para navegador \textit{Chrome}).
    \item \href{https://contrastchecker.com/}{Contrast Checker} (\textit{website} - Validação de contraste de cores).
\end{itemize}
\item Avaliações manuais: Essa etapa se deu usando o leitor de tela e a navegação por teclado, onde foi checado se todos os comportamentos e ações esperados estão sendo realmente respeitados e executados. Quando os testes manuais não atingiam os requisitos de aceitação a equipe de \textit{quality assurance} reportava aos desenvolvedores os pontos de falhas para que fossem corrigidos.
\end{itemize}


}

\subsection{Tecnologias utilizadas} 
{A linguagem de programação JavaScript foi escolhida para a elaboração deste trabalho devido a sua grande relevância no \textit{front-end}. Entre as principais bibliotecas utilizadas neste projeto destaca-se:
\begin{itemize}
\item \textit{\href{https://developer.mozilla.org/pt-BR/docs/Web/HTML}{HTML}}: é uma linguagem de marcação utilizada na construção de páginas na \textit{Web}.
\item Diretrizes \textit{\href{https://www.w3c.br/traducoes/wcag/wcag21-pt-BR/}{WCAG}}: é um conjunto de regras que definem a forma de como tornar o conteúdo da \textit{Web} mais acessível para pessoas com deficiência.
\item \textit{Atributos \href{https://www.w3.org/TR/wai-aria-1.2/}{WAI-ARIA}}: Especificações técnicas que oferecem maneiras de tornar as aplicações mais acessíveis a uma diversidade maior de pessoas, incluindo quem utiliza tecnologias assistivas, como leitores de telas.
\item { \href{https://pt-br.reactjs.org/}{React.js}}: É uma biblioteca \textit{JavaScript} para criar interfaces de usuário, ela é declarativa e baseada em componentes.
\item {\href{https://sass-lang.com/}{Sass}}: Ou \textit{Syntactically Awesome Style Sheets} é uma linguagem de extensão do \textit{CSS} e tem como objetivo tornar o processo de desenvolvimento mais simples e eficiente.
\item {\href{https://d3js.org/}{D3.js}}: Trata-se de uma biblioteca {JavaScript} para construção de visualizações de dados. O \textit{D3} permite que visualizações sejam criadas diretamente em páginas \textit{HTML} através de gráficos vetorizados \lstinline{SVG}.
\item Tecnologias assistivas: Para a construção do \textit{e-commerce} foram usados alguns tipos de leitores de telas que transcrevem para linguagem falada o conteúdo textual presente na tela do dispositivo eletrônico. 
\begin{itemize}
\item \href{https://www.apple.com/br/accessibility/vision/}{VoiceOver}: Leitor de tela presente nos produtos e ambientes virtuais confeccionados pela \textit{Apple}. 
\item \href{https://www.nvaccess.org/download/}{NVDA}: Leitor de telas voltado para ambientes que usam \textit{Windows}
\item \href{https://help.gnome.org/users/orca/stable/index.html.pt_BR}{ORCA}: Leitor de telas voltado para ambientes \textit{Linux}
\end{itemize}
\end{itemize}
 }
 
 
\subsection{Contribuição}
{

 A atuação da autora desse estudo consistiu na construção de soluções de interface que melhorem a experiência do usuário e no desenvolvimento de tais soluções em softwares. Para criar esses softwares aplicando essas soluções foram utilizadas tecnologias de \textit{front-end}. Portanto, a autora esteve presente desde a fase de ideação de \textit{design} até o desenvolvimento do \textit{e-commerce} fazendo o uso de JavaScript e da biblioteca React para a construção do mesmo. A solução apresentada não se encontra em ambiente de produção pois há outros épicos que não englobam acessibilidade que ainda estão sendo construídos dentro do \textit{e-commerce} . Entretanto, todo o épico de acessibilidade e \textit{Data Visualization} foi muito bem avaliado pelo cliente que se mostrou muito feliz com o que foi apresentado e aprovou todo o progresso feito.

O projeto, o guia construído, o \textit{style guide} e o conhecimento adquirido para a construção da solução se tornou um case para outros desenvolvedores e designers que buscam aprender mais sobre a acessibilidade e aplicá-la de um modo descomplicado e correto, sendo apresentado como case em reuniões de troca de conhecimentos entre pessoas desenvolvedoras e designers.  

}
\section{Dificuldades encontradas}
\label{sec:dificuldades}
{
Uma das maiores dificuldades encontradas ao longo do desenvolvimento deste estudo foi o fato de ter que lidar com uma documentação extremamente extensa do WCAG. Como a WCAG são as diretrizes de acessibilidades para vários tipos de deficiências foi preciso ler toda a documentação para depois focar nas diretrizes que deveriam ser aplicadas de fato. Fazer esse filtro foi bem cansativo, pois além de ser muita informação cada diretriz deveria ser testada se ela se aplicava nas necessidades do cliente.

Lidar com sistema legado nunca foi fácil e torná-lo acessível foi verdadeiramente um desafio para o time inteiro, pois havia muitos componentes com zero acessibilidade que deveriam ser adaptados ou refeitos. Essa escolha entre adaptar ou refazer se tornou complicada, pois refazer sempre se torna mais atrativo, mas quando lidamos com prazo de entrega essa escolha deve ser ponderada pelo time inteiro.

O sistema já tinha certos problemas de performance e isso foi uma preocupação para o time quando entramos na parte de DataVis, pois a plotagem de gráfico pode ser tornar algo custoso ao sistema. Para minimizar os problemas de performance foi feita uma varredura no sistema para detectar os pontos de gargalo e no dashboard foi usado o \lstinline{lazy loading} nativo do JavaScript que ajudou em possíveis gargalos ao se plotar muitos gráficos de uma só vez. 
}


\section{Impactos da sua formação no seu trabalho}
\label{sec:impactos}

\textcolor{violet}{Descreva com detalhes como sua formação lhe ajudou nas atividades que você desenvolve na empresa. Podem ser trazidos conhecimentos adquiridos em disciplinas cursadas, em projetos de disciplinas, em iniciação científica, projetos extra-curriculares, etc.}
\section{Conclusão}
\label{sec:conclusao}

\textcolor{violet}{Nesta seção você poderá fazer um síntese da sua atividade profissional na empresa dando destaque as suas principais contribuições e como elas impactaram o conjunto da empresa. Também poderá fazer considerações sobre possíveis atividades futuras que podem ou não ser decorrentes das atividades descritas neste documento.}


\renewcommand\refname{Referências Bibliográficas}
%\bibliographystyle{abntex2-alf}
\bibliographystyle{IEEEtran}
\bibliography{referencias.bib}
%\listoftodos
\addcontentsline{toc}{section}{\numberline{}Referências Bibliográficas}
\renewcommand*{\chaptername}{\appendixname}

% \addcontentsline{toc}{section}{\numberline{}Apêndice A}

\newpage

\appendix
\begin{appendices}


\section{Diretrizes WCAG}
\label{sec:apendice}

O apêndice em questão lista todas as diretrizes WCAG que foram usadas para serem aplicadas no \textit{e-commerce} a fim de diminuir barreiras de acesso para deficientes visuais.

\subsection{Nível A}
{

\noindent\begin{minipage}{\linewidth}
\centering
\captionof{table}{Diretrizes do Nível A}\resizebox{\linewidth}{!}{%
\begin{tabular}{|l|l|p{400px}|c|} 
\hline
\rowcolor[HTML]{ECF4FF} 
{\color[HTML]{333333} Número} & Nome & Descrição & Aplicação no projeto\\ 
\hline
    1.1.1 &
      Conteúdo Não Textual &
      Todo o conteúdo não textual que é exibido ao usuário tem uma alternativa textual que serve a um propósito equivalente, exceto para algumas situações. & sim \\ \hline
    1.3.1 &
      Informações e Relações &
      As estruturas da tela devem ser construída de forma que sua arquitetura de informação faça sentido tanto para todos, sejam ouvintes ou leitores. & sim\\ \hline
    1.3.2 & 
        Sequência com significado &
        A apresentação das informações na tela sempre deverá ter uma sequência lógica. & sim \\\hline
    1.4.1 & 
        Utilização de cores & As cores não devem carregar significado lógico, elas não devem ser utilizadas como única maneira de transmitir conteúdo ou distinguir elementos visuais. & sim \\\hline
    2.1.1 & 
        Teclado & Todas as funcionalidades devem ser acionadas via teclado, com exceção se a funcionalidade não possibilite o controle apenas por teclado. & sim\\\hline
    2.1.2 & 
        Sem bloqueio de teclado & Ao se interagir via teclado, a navegação por todos os elementos ”clicáveis” deve ocorrer sem que haja bloqueios ou interrupções. & sim \\\hline
    2.4.1 & 
        Ignorar blocos & Deve ser fornecido um tipo de controle para que as pessoas possam ignorar determinados conteúdos repetitivos e assim continuar com a navegação. & sim\\\hline
    2.4.2 & 
        Página com título & Todas as telas devem ter um título principal e que descreva claramente a sua finalidade. & sim  \\\hline
    2.4.3 & 
        Ordem do foco & A interação por elementos focáveis na tela sempre deverá ser sequencial e lógica de acordo com o conteúdo apresentado. & sim \\\hline
    2.4.4 & 
        Finalidade do link (em contexto) & A finalidade de um link deve ser determinada a partir do texto do próprio link ou a partir do contexto no entorno deste link. & sim\\\hline
    2.4.7 & 
        Foco visível & Ao se interagir por teclado, qualquer pessoa deve conseguir identificar qual é a sua localização espacial na tela através de um foco visível identificador de sua localização. & sim \\\hline
    2.5.2 & 
        Cancelamento de acionamento & Deve se fornecer um modo de cancelar acionamentos feitos de forma não proposital. & sim\\\hline
    2.5.3 & 
        Rótulo no Nome acessível & Rótulos em botões, ícones acionáveis ou qualquer controle interativo, devem ter uma descrição significativa. & sim \\\hline
    3.1.1 & 
        Idioma da página & Declarar adequadamente o idioma da tela faz com que leitores de telas utilizem uma entonação correta para citar conteúdos. Sempre os declare. & sim\\\hline
    3.2.1 & 
        Em foco & O foco sempre deve se manter durante a navegação, sempre evitar mudança contextual que possa desorientar alguém. & sim \\\hline
    3.3.1 & 
        Identificação do erro & Sempre que uma mensagem de erro for exibida, ela deve identificar claramente qual é o elemento que gerou o erro de forma visual e audível. & sim\\\hline
    3.3.2 & 
        Rótulos e instruções & Todos os rótulos devem descrever claramente e sem ambiguidades a finalidade dos campos de formulário. & sim\\\hline
    4.1.1 & 
        Análise & Deve ser fornecido código semanticamente correto e sem erros significativos. & sim\\\hline
    4.1.2 & 
        Nome, função, valor & Toda tecnologia assistiva faz uso das propriedades de nome, função e valor para identificar adequadamente os elementos padronizados do HTML. Qualquer componente customizado deve trazer também essas marcações de forma adequada. 
        & sim\\ \hline
\end{tabular}
}
\label{Diretrizes nível A}

\end{minipage}

}

\newpage

\subsection{Nível AA}
{

\noindent\begin{minipage}{\linewidth}
\centering
\captionof{table}{Diretrizes do Nível AA}\resizebox{\linewidth}{!}{%
\begin{tabular}{|l|l|p{400px}|c|} 
\hline
\rowcolor[HTML]{ECF4FF} 
{\color[HTML]{333333} Número} & Nome & Descrição & Aplicação no projeto\\ 
\hline
    1.3.5 &
       Identificar o objetivo de entrada & Deve ser claro para as pessoas o que deve ser preenchido em campos de formulários. & sim \\ \hline
    1.4.3 & 
        Contraste (mínimo) & Textos devem ter uma relação de contraste entre primeiro e segundo plano de ao menos 4.5:1. & sim\\ \hline
    1.4.4 &
        Redimensionar texto & Ao se aplicar zoom de até 200 \%  na tela, deve ocorrer a responsividade dos textos apresentados de forma que sua leitura e legibilidade continuem adequados sem qualquer quebra na apresentação das informações. & sim\\ \hline
    1.4.11 & 
        Contraste Não-Textual & Componentes de interface e imagens essenciais para o entendimento do conteúdo devem ter uma relação de contraste entre primeiro e segundo plano de ao menos 3:1. & sim\\ \hline
    1.4.12 &
        Espaçamento de texto & Sempre que houver um redimensionamento os textos não devem perder legibilidade. & sim \\ \hline
    1.4.13 &
        Conteúdo em foco por mouse ou teclado &  Conteúdos adicionais não devem ser acionados apenas com foco por mouse ou teclado. & sim\\ \hline
    2.4.5 &
        Várias formas &  Deve ser fornecido mais de uma forma de as pessoas encontrarem um determinado conteúdo. & sim \\ \hline
    2.4.6 &
        Cabeçalhos e rótulos & Todos os títulos e rótulo devem descrever claramente a finalidade dos conteúdos, não deve haver ambiguidade em seu entendimento. & sim\\ \hline
    3.1.2 &
        Idioma das partes & O idioma de uma determinada palavra ou frase contendo idioma diferente do original da tela, deve ser definido e corretamente identificado para que também ocorra uma correta entonação e pronúncia adequada via leitores de tela. & sim\\ \hline
    3.2.3 &
        Navegação consistente & Deve-se manter a consistência com relação ao formato de apresentação, interação e localização na tela. & sim\\ \hline
    3.2.4 &
        Identificação consistente & Deve-se manter a consistência com relação a diferentes formatos de elementos, mas que possuem uma mesma funcionalidade. & não\\ \hline
    3.3.3  &
        Sugestão de erro & Sempre que uma mensagem de erro for exibida, ela deve também dar dicas de como resolver o erro. & sim\\ \hline
    3.3.4  &
        Prevenção de erro (legal, financeiro, dados) & Deve ser fornecida uma forma de confirmação de dados ou a possibilidade de cancelamento do envio, sempre que campos de formulários exigirem o preenchimento de dados que envolvam responsabilidade jurídica, financeira ou contenham dados sensíveis. & sim\\ \hline
    4.1.3  &
         Mensagens de status & Qualquer tipo de mensagem que é resultado de uma ação ou que informa o andamento de um processo e que seja relevante para a pessoa, deve ser transmitida sem que ocorra uma mudança de foco na tela. & sim\\ \hline

\end{tabular}
}
\label{Diretrizes nível AA}

\end{minipage}

}

\subsection{Nível AAA}
{

\noindent\begin{minipage}{\linewidth}
\centering
\captionof{table}{Diretrizes do Nível AAA}\resizebox{\linewidth}{!}{%
\begin{tabular}{|l|l|p{400px}|c|} 
\hline
\rowcolor[HTML]{ECF4FF} 
{\color[HTML]{333333} Número} & Nome & Descrição & Aplicação no projeto\\ 
\hline
    1.4.6  &
       Contraste (melhorado) & Textos devem ter uma relação de contraste entre primeiro e segundo plano de ao menos 7:1. & sim\\ \hline
    2.1.3  &
       Teclado (sem exceção) & Todas as funcionalidades devem ser acionadas via teclado, sem exceção. & sim\\ \hline
    2.2.3  &
       Sem limite de tempo & Nenhuma funcionalidade em tela deve possuir algum tipo de execução mediante o cumprimento em um determinado período de tempo. & não\\ \hline
    2.2.4  &
      Interrupções & Deve ser possível adiar os desligar qualquer tipo de interrupção acionada no sistema. & não\\ \hline
    2.2.5  &
      Nova autenticação & Quando uma sessão autenticada expira, o usuário deve ser capaz de continuar sua atividade sem que haja perca de dados até que seja feita um nova autenticação. & sim\\ \hline
    2.4.9  &
      Finalidade do link (apenas link) & A finalidade de um link deve ser determinada a partir do texto do próprio link. & sim\\ \hline
    2.4.10  &
      Cabeçalhos da seção & Sempre que o conteúdo da tela for dividido em sessões, todas devem possuir títulos claros, com níveis de hierarquia bem definidos, facilitando a identificação das áreas. & sim\\ \hline
    2.4.8  &
      Localização & Qualquer pessoa deve conseguir se localizar ou se orientar facilmente em qualquer nas telas. & sim\\ \hline
    2.5.5  &
      Tamanho da área clicável & O tamanho das áreas acionáveis por clique ou toque devem possuir no mínimo 44x44 pixeis de espaçamento, a não ser quando essa área esteja em uma frase localizada em um bloco de texto. & sim\\ \hline
    3.1.3 &
      Palavras incomuns & O uso de gírias, jargões, metáforas e figuras de linguagem pode ser um empecilho para a compreensão da informação, nesse sentido deve-se fornecer uma forma de tradução ou explicação da informação. & não\\ \hline
    3.1.4 &
      Abreviações & Nem sempre uma abreviação ou um acrônimo é compreensível por todas as pessoas, nesse sentido deve-se fornecer uma forma de identificação de seu significado real. & não \\ \hline
    3.1.6 &
      Pronúncia & Palavras regionais específicas e nomes próprios costumam ter pronúncias também específicas. Deve ser fornecida uma forma de possibilitar a correta compreensão da pronúncia em alguns casos. & não\\ \hline
    3.2.5 &
      Alteração a pedido & Qualquer mudança de contexto que possa desorientar as pessoas, deve ocorrer apenas quando solicitada pela pessoa que está utilizando. & sim\\ \hline
    3.3.6 &
      Prevenção de erro (todos) & Deve ser fornecida uma forma de confirmação de dados ou a possibilidade de cancelamento do envio, sempre que campos de formulários exigirem o preenchimento de dados. & sim\\ \hline

\end{tabular}
}
\label{Diretrizes nível AAA}

\end{minipage}

}


\end{appendices}

\end{document}