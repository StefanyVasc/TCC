\section{Dificuldades encontradas}
\label{sec:dificuldades}
{
Uma das maiores dificuldades encontradas ao longo do desenvolvimento deste estudo foi o fato de ter que lidar com uma documentação extremamente extensa do WCAG. Como a WCAG são as diretrizes de acessibilidades para vários tipos de deficiências foi preciso ler toda a documentação para depois focar nas diretrizes que deveriam ser aplicadas de fato. Fazer esse exigiu muito esforço, pois além de ter muita informação em cada diretriz deveria ser feita uma avaliação para ter a certeza se a diretriz em questão se aplicava nas necessidades e requisitos levantados pelo cliente.

Lidar com sistema legado nunca foi fácil e torná-lo acessível foi verdadeiramente um desafio para o time inteiro, pois havia muitos componentes inacessíveis que deveriam ser adaptados ou refeitos. Essa escolha entre adaptar ou refazer se tornou complicada, pois refazer sempre se torna mais atrativo, mas quando lidamos com prazo de entrega essa escolha deve ser ponderada pelo time inteiro.

O sistema já tinha certos problemas de performance e isso foi uma preocupação para o time quando entramos na parte de DataVis, pois a plotagem de gráfico pode ser tornar algo custoso ao sistema. Para minimizar os problemas de performance foi feita uma varredura no sistema para detectar os pontos de gargalo e no dashboard foi usado o \lstinline{lazy loading} nativo do JavaScript que ajudou em possíveis gargalos ao se plotar muitos gráficos de uma só vez. 
}

