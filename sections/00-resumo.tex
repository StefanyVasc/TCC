
{O surgimento do \textit{e-commerce} modificou a forma como as pessoas fazem compras e como consequência disso houve um aumento significativo no número de lojas \textit{on-line}. Um estudo recente publicado pela Accenture mostrou que pessoas com deficiência possuem poder aquisitivo e sentem vontade de efetuar compras e atrelado às suas limitações as lojas \textit{on-line} se mostram como sendo uma ótima opção para efetuar suas compras sem ter a necessidade de sair de casa. O problema surge quando essas lojas \textit{on-line} não se mostram preparadas o suficiente para suprir as necessidades das pessoas com deficiências, especificamente deficiência visual, e como resultado desse despreparo ocorre a marginalização desse público-alvo, fato que fere os direitos básicos desses indivíduos. Utilizando os conhecimentos adquiridos por meio de um estudo focado sobre diretrizes Web Content Acessibility Guidelines (WCAG) e padrões de acessibilidade na internet foi possível fazer a construção de alguns artefatos que serviram como guias para tomadas de decisões e padrões de implementações. Neste \textit{guideline} estão presentes técnicas de desenvolvimento de \textit{software} e design inclusivo.
\vspace{1,5cm}
\newline
\textbf{Palavras-chave:} acessibilidade; e-commerce; style guide; diretrizes WCAG; atributos WAI-ARIA; gráficos acessíveis.
} 



\vspace{60px}
{

\centerline{\textbf{Abstract}}
% \newline
\vspace{10px}
The emergence of e-commerce has changed the way people shop and as a result there has been a significant increase in the number of online stores. A recent study published by Accenture showed that people with disabilities have purchasing power and feel like shopping, and linked to their limitations, online stores are shown to be a great option to make their purchases without having to leave their homes. House. The problem arises when these online stores are not prepared enough to meet the needs of people with disabilities, specifically visual impairment, and as a result of this unpreparedness, this target audience is marginalized, a fact that violates the basic rights of these individuals. Using the knowledge acquired through a study
Based on Web Content Accessibility Guidelines (WCAG) and internet accessibility standards, it was possible to build some artifacts that served as guides for decision-making and implementation standards. In this guideline are present techniques of software development and inclusive design.
\vspace{1,5cm}
\newline
\textbf{Keywords:} accessibility; e-commerce; styleguide; WCAG guidelines; WAI-ARIA attributes; accessible graphics.

}