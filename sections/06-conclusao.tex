\section{Conclusão}
\label{sec:conclusao}


{

A praticidade de conseguir efetuar compras sem sair de casa é algo que vem sendo amplamente explorado por consumidores e comerciantes. Como foi visto ao longo desse trabalho, essa praticidade não se aplica para todos os tipos de consumidores. Existe uma parcela da população que não consegue fazer suas compras, mesmo esse sendo um direito conferido por lei.

Pela internet ser algo tão inerente e necessária à vida humana, a experiência com o seu uso deve ser semelhante para todas as pessoas, sejam elas deficientes ao não. Para promover essa semelhança de experiência foi necessário repensar o \textit{e-commerce}
nos mínimos detalhes levando em conta a própria complexidade de construção sem ter a acessibilidade devida.

Um ponto levado em consideração neste trabalho foi que dificilmente um site consegue ser 100\% acessível, pois muitas das diretrizes se sobrepõem ou desfazem a construção de outra. Sendo assim, deve-se trabalhar com a maximização de quebra de barreiras de acesso. No caso do \textit{e-commerce} construído e detalhado nesse trabalho o foco das quebras barreiras se deu para com as pessoas que possuem alguma limitação ou deficiência visual. 

Toda a construção desse trabalho converge em prover abordagens sobre a falta de acessibilidade em lojas \textit{on-line}. Logo, o principal objetivo foi transformação de um \textit{e-commerce} não acessível em acessível e para alcança-lo durante o processo de desenvolvimento ficou claro a necessidade de documentos que apoiassem a pessoa desenvolvedora durante a construção, visto que acessibilidade é um ponto que acaba sendo deixado de lado pois não há muita procura partindo da comunidade de desenvolvedores. Esses documentos especificavam decisões de design e de comportamento diante da \textit{interface} do usuário e se mostraram excepcionais como apoio para tomadas de decisões, esclarecimento de dúvidas além de ser o guia de padrões para a construção da UI.

Outro ponto de extrema importância presente nesse trabalho e pouco difundido diante dos padrões de acessibilidade e desenvolvimento está na construção de gráficos que são acessíveis e navegáveis. O objetivo principal de um gráfico é trazer a visualização da combinação de dados de forma facilitada e para isso se faz uso de cores e em termos de acessibilidade o maior desafio é conseguir construir o mesmo tipo de visualização facilitada para as pessoas que possuem limitações visuais. 

A visualização facilitada entregue por um gráfico é um recurso muito importante no entendimento geral do conteúdo daquele gráfico e as pessoas que são deficientes visuais devem estar cientes de tudo o que se passa no gráfico e em seus dados. Para eliminar algumas barreiras de acesso quanto a essa visualização foi mostrada algumas técnicas de construção de gráficos de forma acessível utilizando a biblioteca D3.js. As barreiras de acesso mais abordadas quanto a gráficos foram a dificuldade de navegação que os usuário enfrentam e a falta de rotulação para cada parte que o gráfico dispõe, pois só não há como navegar por algo que não é lido pelo leitor de telas.

Mesmo tendo todas essas abordagens para diminuir as barreiras encontradas diante de \textit{Data Visualization} existem casos complexos que apenas o fornecimento dos dados de uma forma que não seja por meio de gráficos será clara o suficiente para se tornar entendível para uma pessoas com deficiência visual.

Por vezes a acessibilidade passa despercebido por uma pessoa que não tem deficiência, mas esse panorama vem sendo modificado aos poucos. A acessibilidade deixou de ser um tema apenas difundido por pessoas com deficiência e começou a se tornar um tema costumeiro. A acessibilidade vem muito antes de transformar um \textit{wireframe} em um \textit{website}, ela vem desde a fase de planejamento.


Hoje em dia existem tecnologias e padrões de apoio para os desenvolvedores se empenharem em tornar a \textit{web} acessível para todos. Contudo, há um débito comportamental por parte dos desenvolvedores de \textit{software} uma vez que em algum momento do desenvolvimento não foi considerado os diferentes tipos de pessoas e necessidades. A \textit{web} acessível depende daqueles que a constroem, acessibilidade não é opção é um direito e deve ser preservado por todos.  


No que se refere às melhorias e trabalhos futuros, há alguns pontos a serem implementados de modo a complementar a acessibilidade e usabilidade do \textit{e-commerce}: 
\begin{enumerate}
    \item Realizar testes com os \textit{stakeholders} e levantar pontos de melhorias mais focados nas necessidades de deficientes visuais. 
    \item Possibilidade da criação de um reconhecimento de padrões por meio de aprendizado de máquina para sugerir itens de compras de acordo com os gostos do usuário. 
    \item Implementação de resumos de visualização de dados de acordo com os filtros mais usados por aquele usuário, onde esses resumos devem apresentar duas formas de visualização: em texto corrido com a sua estrutura navegável e a outra em um documento pdf. 
    \item Implementação de filtros preferenciais e pré-programados no dashboard. 
\end{enumerate}
}