
\newpage

\appendix
\begin{appendices}


\section{Diretrizes WCAG}
\label{sec:apendice}

O apêndice em questão lista todas as diretrizes WCAG que foram usadas para serem aplicadas no \textit{e-commerce} afim de diminuir barreiras de acesso para deficientes visuais.

\subsection{Nível A}
{

\noindent\begin{minipage}{\linewidth}
\centering
\captionof{table}{Diretrizes do Nível A}\resizebox{\linewidth}{!}{%
\begin{tabular}{|l|l|p{400px}|c|} 
\hline
\rowcolor[HTML]{ECF4FF} 
{\color[HTML]{333333} Número} & Nome & Descrição & Aplicação no projeto\\ 
\hline
    1.1.1 &
      Conteúdo Não Textual &
      Todo o conteúdo não textual que é exibido ao usuário tem uma alternativa textual que serve a um propósito equivalente, exceto para algumas situações. & sim \\ \hline
    1.3.1 &
      Informações e Relações &
      As estruturas da tela devem ser construída de forma que sua arquitetura de informação faça sentido tanto para todos, sejam ouvintes ou leitores. & sim\\ \hline
    1.3.2 & 
        Sequência com significado &
        A apresentação das informações na tela sempre deverá ter uma sequência lógica. & sim \\\hline
    1.4.1 & 
        Utilização de cores & As cores não devem carregar significado lógico, elas não devem ser utilizadas como única maneira de transmitir conteúdo ou distinguir elementos visuais. & sim \\\hline
    2.1.1 & 
        Teclado & Todas as funcionalidades devem ser acionadas via teclado, com exceção se a funcionalidade não possibilite o controle apenas por teclado. & sim\\\hline
    2.1.2 & 
        Sem bloqueio de teclado & Ao se interagir via teclado, a navegação por todos os elementos ”clicáveis” deve ocorrer sem que haja bloqueios ou interrupções. & sim \\\hline
    2.4.1 & 
        Ignorar blocos & Deve ser fornecido um tipo de controle para que as pessoas possam ignorar determinados conteúdos repetitivos e assim continuar com a navegação. & sim\\\hline
    2.4.2 & 
        Página com título & Todas as telas devem ter um título principal e que descreva claramente a sua finalidade. & sim  \\\hline
    2.4.3 & 
        Ordem do foco & A interação por elementos focáveis na tela sempre deverá ser sequencial e lógica de acordo com o conteúdo apresentado. & sim \\\hline
    2.4.4 & 
        Finalidade do link (em contexto) & A finalidade de um link deve ser determinada a partir do texto do próprio link ou a partir do contexto no entorno deste link. & sim\\\hline
    2.4.7 & 
        Foco visível & Ao se interagir por teclado, qualquer pessoa deve conseguir identificar qual é a sua localização espacial na tela através de um foco visível identificador de sua localização. & sim \\\hline
    2.5.2 & 
        Cancelamento de acionamento & Deve se fornecer um modo de cancelar acionamentos feitos de forma não proposital. & sim\\\hline
    2.5.3 & 
        Rótulo no Nome acessível & Rótulos em botões, ícones acionáveis ou qualquer controle interativo, devem ter uma descrição significativa. & sim \\\hline
    3.1.1 & 
        Idioma da página & Declarar adequadamente o idioma da tela faz com que leitores de telas utilizem uma entonação correta para citar conteúdos. Sempre os declare. & sim\\\hline
    3.2.1 & 
        Em foco & O foco sempre deve se manter durante a navegação, sempre evitar mudança contextual que possa desorientar alguém. & sim \\\hline
    3.3.1 & 
        Identificação do erro & Sempre que uma mensagem de erro for exibida, ela deve identificar claramente qual é o elemento que gerou o erro de forma visual e audível. & sim\\\hline
    3.3.2 & 
        Rótulos e instruções & Todos os rótulos devem descrever claramente e sem ambiguidades a finalidade dos campos de formulário. & sim\\\hline
    4.1.1 & 
        Análise & Deve ser fornecido código semanticamente correto e sem erros significativos. & sim\\\hline
    4.1.2 & 
        Nome, função, valor & Toda tecnologia assistiva faz uso das propriedades de nome, função e valor para identificar adequadamente os elementos padronizados do HTML. Qualquer componente customizado deve trazer também essas marcações de forma adequada. 
        & sim\\ \hline
\end{tabular}
}
\label{Diretrizes nível A}

\end{minipage}

}

\newpage

\subsection{Nível AA}
{

\noindent\begin{minipage}{\linewidth}
\centering
\captionof{table}{Diretrizes do Nível AA}\resizebox{\linewidth}{!}{%
\begin{tabular}{|l|l|p{400px}|c|} 
\hline
\rowcolor[HTML]{ECF4FF} 
{\color[HTML]{333333} Número} & Nome & Descrição & Aplicação no projeto\\ 
\hline
    1.3.5 &
       Identificar o objetivo de entrada & Deve ser claro para as pessoas o que deve ser preenchido em campos de formulários. & sim \\ \hline
    1.4.3 & 
        Contraste (mínimo) & Textos devem ter uma relação de contraste entre primeiro e segundo plano de ao menos 4.5:1. & sim\\ \hline
    1.4.4 &
        Redimensionar texto & Ao se aplicar zoom de até 200 \%  na tela, deve ocorrer a responsividade dos textos apresentados de forma que sua leitura e legibilidade continuem adequados sem qualquer quebra na apresentação das informações. & sim\\ \hline
    1.4.11 & 
        Contraste Não-Textual & Componentes de interface e imagens essenciais para o entendimento do conteúdo devem ter uma relação de contraste entre primeiro e segundo plano de ao menos 3:1. & sim\\ \hline
    1.4.12 &
        Espaçamento de texto & Sempre que houver um redimensionamento os textos não devem perder legibilidade. & sim \\ \hline
    1.4.13 &
        Conteúdo em foco por mouse ou teclado &  Conteúdos adicionais não devem ser acionados apenas com foco por mouse ou teclado. & sim\\ \hline
    2.4.5 &
        Várias formas &  Deve ser fornecido mais de uma forma de as pessoas encontrarem um determinado conteúdo. & sim \\ \hline
    2.4.6 &
        Cabeçalhos e rótulos & Todos os títulos e rótulo devem descrever claramente a finalidade dos conteúdos, não deve haver ambiguidade em seu entendimento. & sim\\ \hline
    3.1.2 &
        Idioma das partes & O idioma de uma determinada palavra ou frase contendo idioma diferente do original da tela, deve ser definido e corretamente identificado para que também ocorra uma correta entonação e pronúncia adequada via leitores de tela. & sim\\ \hline
    3.2.3 &
        Navegação consistente & Deve-se manter a consistência com relação ao formato de apresentação, interação e localização na tela. & sim\\ \hline
    3.2.4 &
        Identificação consistente & Deve-se manter a consistência com relação a diferentes formatos de elementos, mas que possuem uma mesma funcionalidade. & não\\ \hline
    3.3.3  &
        Sugestão de erro & Sempre que uma mensagem de erro for exibida, ela deve também dar dicas de como resolver o erro. & sim\\ \hline
    3.3.4  &
        Prevenção de erro (legal, financeiro, dados) & Deve ser fornecida uma forma de confirmação de dados ou a possibilidade de cancelamento do envio, sempre que campos de formulários exigirem o preenchimento de dados que envolvam responsabilidade jurídica, financeira ou contenham dados sensíveis. & sim\\ \hline
    4.1.3  &
         Mensagens de status & Qualquer tipo de mensagem que é resultado de uma ação ou que informa o andamento de um processo e que seja relevante para a pessoa, deve ser transmitida sem que ocorra uma mudança de foco na tela. & sim\\ \hline

\end{tabular}
}
\label{Diretrizes nível AA}

\end{minipage}

}

\subsection{Nível AAA}
{

\noindent\begin{minipage}{\linewidth}
\centering
\captionof{table}{Diretrizes do Nível AAA}\resizebox{\linewidth}{!}{%
\begin{tabular}{|l|l|p{400px}|c|} 
\hline
\rowcolor[HTML]{ECF4FF} 
{\color[HTML]{333333} Número} & Nome & Descrição & Aplicação no projeto\\ 
\hline
    1.4.6  &
       Contraste (melhorado) & Textos devem ter uma relação de contraste entre primeiro e segundo plano de ao menos 7:1. & sim\\ \hline
    2.1.3  &
       Teclado (sem exceção) & Todas as funcionalidades devem ser acionadas via teclado, sem exceção. & sim\\ \hline
    2.2.3  &
       Sem limite de tempo & Nenhuma funcionalidade em tela deve possuir algum tipo de execução mediante o cumprimento em um determinado período de tempo. & não\\ \hline
    2.2.4  &
      Interrupções & Deve ser possível adiar os desligar qualquer tipo de interrupção acionada no sistema. & não\\ \hline
    2.2.5  &
      Nova autenticação & Quando uma sessão autenticada expira, o usuário deve ser capaz de continuar sua atividade sem que haja perca de dados até que seja feita um nova autenticação. & sim\\ \hline
    2.4.9  &
      Finalidade do link (apenas link) & A finalidade de um link deve ser determinada a partir do texto do próprio link. & sim\\ \hline
    2.4.10  &
      Cabeçalhos da seção & Sempre que o conteúdo da tela for dividido em sessões, todas devem possuir títulos claros, com níveis de hierarquia bem definidos, facilitando a identificação das áreas. & sim\\ \hline
    2.4.8  &
      Localização & Qualquer pessoa deve conseguir se localizar ou se orientar facilmente em qualquer nas telas. & sim\\ \hline
    2.5.5  &
      Tamanho da área clicável & O tamanho das áreas acionáveis por clique ou toque devem possuir no mínimo 44x44 pixeis de espaçamento, a não ser quando essa área esteja em uma frase localizada em um bloco de texto. & sim\\ \hline
    3.1.3 &
      Palavras incomuns & O uso de gírias, jargões, metáforas e figuras de linguagem pode ser um empecilho para a compreensão da informação, nesse sentido deve-se fornecer uma forma de tradução ou explicação da informação. & não\\ \hline
    3.1.4 &
      Abreviações & Nem sempre uma abreviação ou um acrônimo é compreensível por todas as pessoas, nesse sentido deve-se fornecer uma forma de identificação de seu significado real. & não \\ \hline
    3.1.6 &
      Pronúncia & Palavras regionais específicas e nomes próprios costumam ter pronúncias também específicas. Deve ser fornecida uma forma de possibilitar a correta compreensão da pronúncia em alguns casos. & não\\ \hline
    3.2.5 &
      Alteração a pedido & Qualquer mudança de contexto que possa desorientar as pessoas, deve ocorrer apenas quando solicitada pela pessoa que está utilizando. & sim\\ \hline
    3.3.6 &
      Prevenção de erro (todos) & Deve ser fornecida uma forma de confirmação de dados ou a possibilidade de cancelamento do envio, sempre que campos de formulários exigirem o preenchimento de dados. & sim\\ \hline

\end{tabular}
}
\label{Diretrizes nível AAA}

\end{minipage}

}


\end{appendices}