\section{Impactos da sua formação no seu trabalho}
\label{sec:impactos}

{
O curso de Bacharelado em Sistemas de Informação busca formar os alunos com um perfil multidisciplinar sempre com o foco para o mercado de TI. Nesta seção são exploradas as disciplinas curriculares do curso que foram de extrema importância na vivência profissional.

Metodologia ágil é algo bem forte dentro do mercado e uma das formas mais eficientes para se trabalhar com entregas. Os conhecimentos adquiridos na disciplina de gerência de projetos de software sobre entregas com \textit{sprints} semanais,  reuniões como \textit{daily}, \textit{planning} e \textit{sprint review} foram essenciais para entender e aplicar essa forma de trabalho.

A linguagem de programação JavaScript, abordada na disciplina de desenvolvimento de aplicações para \textit{web}, é essencial para qualquer solução que sejam desenvolvida para a camada do \textit{frontend}. Juntamente com todos os conhecimentos adquiridos na disciplina de Interfaces Homem-Máquina foram essenciais para entender as necessidades de se construir uma interface que faça sentido para todos os usuários e com isso construir uma solução eficiente tanto na parte visual quanto na parte de eficiência de código.  

Todos as disciplinas que contaram com trabalhos feitos em grupos foram de extrema importância para despertar a consciência e responsabilidade por prazos e entregas. Por estar em grupo ficou muito evidente a importância de se manter a comunicação para a saúde do projeto e esses pontos foram essenciais na construção da postura profissional.
}