\section{A empresa e sua atuação }
\label{sec:atuacao}

{O Centro de Estudos e Sistemas Avançados do Recife, CESAR, foi fundado em 1996 com o proposito de aproximar a academia ao mercado, promovendo iniciativas conjuntas para o desenvolvimento de novos produtos e serviços. No ano 2000 ele foi para o Porto Digital e nos anos seguintes só prosperou. O CESAR está presente dentro do Brasil, tendo Recife como sede e filiais em Manaus, Rio de Janeiro, Sorocaba e Curitiba e também nos Estados Unidos, particularmente na Flórida, onde fica o escritório comercial para contemplar clientes e projetos norte-americanos. 

Como lema "Pessoas impulsionando inovação e inovação impulsionando negócios", a organização conta com um time diverso e multidisciplinar de mais de mil e trezentos colaboradores, incluindo designers, desenvolvedores, consultores, estrategistas, empreendedores, pesquisadores e educadores. Em 2022, o CESAR obteve o certificado Great Place to Work, que trata-se de uma consultoria global que apoia organizações a obter melhores resultados por meio de uma cultura de confiança, alto desempenho e inovação. Essa certificação se reflete em outros números: eNPS de 91.3\%\, que se traduz como índice de satisfação dos colaboradores.

Mais de duas décadas depois da sua criação, a relevância dessa organização se traduz em números: no ano de 2021, o CESAR obteve uma receita de R\$\ 306 milhões, um crescimento de cerca de 60\%\ em relação aos R\$\ 186 milhões de 2020. No ano de 2021 foram mais de 140 projetos executados, mais de 20 segmentos atendidos, mais de 80 clientes atendidos por ano.

Atualmente o CESAR atua em todo o ciclo de inovação, desde pesquisa, formação de pessoas, experimentação e criação de novos modelos de negócios, até o desenvolvimento de soluções robustas e escaláveis. Como principais diferenciais a organização conta com uma escola de inovação, a CESAR School; Uso de Design-Driven, onde todas as soluções criadas são centradas nas pessoas através de abordagens de experimentação; Vasta experiência no desenvolvimento de soluções de alta performance o que garante entregas de classe global; Rede de inovação aberta, onde sempre é buscado parcerias com academia, mercado, empresas maduras e startups. 

Durante esses 25 anos muitas empresas de grande porte buscaram o CESAR para firmaram parcerias e desenvolverem soluções dentre eles: HP, Motorola, Grupo Boticário, Fundação Telefônica Vivo, Petrobras, Globo, Grupo Neoenergia e etc. 

A atuação da autora desse estudo consiste em construir soluções de interface que melhorem a experiência do usuário e converter tais soluções em softwares. Para criar esses softwares aplicando essas soluções são utilizadas tecnologias de front-end. 
}