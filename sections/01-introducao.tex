\newpage
\section{Introdução}
\label{sec:introducao}
 
{Com a pandemia da COVID-19 houve várias mudanças no cenário econômico brasileiro, lojas físicas tiveram que fechar as portas enquanto as lojas virtuais, mais conhecidas como \textit{\textit{e-commerce}}, tiveram um crescimento muito acentuado. Uma pesquisa feita pelo BigData Corp em parceria com o PayPal Brasil \cite{ECOMMERCE} afirma que no ano de 2021 houve um aumento de 22,05\% no número de lojas \textit{on-line}, o que marca cerca de 1,6 milhões de sites de \textit{e-commerce}. Esse aumento de lojas \textit{on-line} refletem o hábito de consumo dos indivíduos, que sempre buscam uma forma de realizar suas compras de forma ágil e descomplicada. Porém existe uma parcela da população que não conseguem efetuar compras em lojas \textit{on-line}, pois tais lojas não estão preparadas para lidar com as necessidades desse público.


}
\subsection{Contexto}
{

De acordo com a  Lei Brasileira de Inclusão Da Pessoa com Deficiência \cite{brasil2015}, 
considera-se pessoa com deficiência aquela que tem impedimento de longo prazo de natureza física, mental, intelectual ou sensorial, o qual, em interação com uma ou mais barreiras, pode obstruir sua participação plena e efetiva na sociedade em igualdade de condições com as demais pessoas.

Ainda no ano de 2019, foi realizada uma pesquisa para entender melhor um pouco sobre o indivíduos com deficiência e segundo a Pesquisa Nacional de Saúde \cite{PNS} existiam quase 7 milhões de pessoas com deficiência visual no Brasil onde aproximadamente 4,2\% eram pessoas com 18 anos ou mais. Para entender um pouco mais sobre esse nicho de consumidores a Accenture produziu um relatório \cite{ACCENTURE} sobre Design Inclusivo \cite{INCLUSIVE-DESIGN}, esse relatório mostra as vantagens de se aplicar o design inclusivo a um negócio. Os dados do relatório demonstraram que no Brasil as pessoas com deficiência possuem uma renda disponível de 5,3 milhões de dólares. Ou seja, essa parcela da população apresenta um poder aquisitivo disponível para efetuar compras, mas muitas empresas ainda não pensaram ou se preocuparam em criar alternativas para vender a estas pessoas. 

Como as lojas \textit{on-line} não estão preparadas para lidar com essa parcela da população, um grande público acaba sendo marginalizado. Esse ato acaba ferindo um dos direitos básicos conferido de acordo com a lei brasileira de inclusão da pessoa com deficiência \cite{brasil2015}, o que torna obrigatória a acessibilidade em ambientes virtuais. Porém, mesma havendo uma legislação, isso não foi suficiente para que houvesse uma mudança dentro do ambiente \textit{Web}. 
}
\subsubsection{Acessibilidade \textit{Web}}
{
Acessibilidade “é a possibilidade de qualquer pessoa usufruir todos os benefícios da sociedade, inclusive o de usar a Internet” \cite{Ferreira_2015}. A acessibilidade na \textit{Web} possibilita que pessoas com deficiência possam navegar livremente na \textit{internet} sem que haja interrupções ou bloqueios. Porém, tornar esse conteúdo acessível pode ser benéfico para um outro grupo de usuários, que podem não ter deficiência mas utilizam a aplicação de um modo diferente, como no caso das pessoas que possuem limitações de visão. 

Quando um sistema é considerado acessível não há barreiras que impeçam o acesso dos usuários. O presente trabalho foi focado na diminuição de barreiras de acesso a pessoas com deficiências e limitações visuais. As necessidades que uma pessoa com deficiência visual tem ao navegar pelos sites são semelhantes a de uma pessoa sem essa deficiência: se atualizar com as últimas notícias, se entreter através de jogos, realizar operações bancárias, fazer compras, etc. A acessibilidade é um dos fatores que compõem a qualidade de um site, por isso a preocupação de diversas instituições e governos em relação a este assunto \cite{sbqs}.

O processo de tornar um site ou sistema acessível deixou de ser tema exclusivo do universo das pessoas com deficiência e começou a se tornar um tema recorrente. O engajamento das pessoas para diminuir as barreiras de acesso é o fator mais importante para iniciar o processo de acessibilização. Dado esse inicio é necessário muito empenho para promover avaliações, correções e melhorias. Vale ressaltar que dificilmente um site ou aplicação será 100\% acessível, uma vez que existem muitos tipos diferentes de deficiências com peculiaridades a seu modo, porém a eliminação de barreiras é algo que pode ser tangível e mensurável. 
}
 

\subsection{Problema e solução proposta}
{Diante da problemática, onde deficientes visuais, mesmo tendo seus direitos garantidos por lei \cite{brasil2015} e sendo consumidores em potencial, acabam sendo negligenciados por empresas que optam por estender seu nicho de mercado para o  meio \textit{on-line}. Este trabalho tem como o principal objetivo propor \textit{guidelines} para diminuir as barreiras de acesso e transformar um \textit{e-commerce} acessível para pessoas com deficiência visual. Para que esse objetivo maior fosse satisfeito foi necessário passar por algumas etapas de construção e estudos, que são elas:   

\begin{itemize}
    \item A sumarização do conhecimento adquirido após o entendimento de como um deficiente visual interage com softwares, seja por navegação via teclado ou com o auxilio de leitores de tela e a documentação desse conhecimento em um \textit{Style Guide} \cite{lynch2016web} que auxilia as pessoas desenvolvedoras nas tomadas de decisões e padrões necessários para a construção de um interface acessível. 
    \item Entendimento das diretrizes Web Content Acessibility Guidelines (WCAG) \cite{WCAG21} para que houvesse um isolamento dessas diretrizes por prioridades e por fim a construção de uma documentação que contava com formas de aplicação eficiente e abordagens já difundidas pela internet.  
    \item Entendimento do \textit{e-commerce} para que fosse possível a aplicação das melhores técnicas de acessibilidade a fim de obter a adaptação de sistemas antigos e sem acessibilidade em sistemas acessíveis.  
    \item Aplicação de técnicas de acessibilidade em elementos gráficos complexos para torná-los identificáveis via navegação por teclado e por leitores de tela. 


\end{itemize}
 
 Desse modo, o presente estudo apresenta um relato de abordagens eficientes aplicadas ao \textit{e-commerce}, sendo baseadas em um guia de acessibilidade construído com o propósito de facilitar o entendimento de diretrizes WCAG e um \textit{style guide} para nortear os desenvolvedores quanto a comportamentos esperados e que faziam sentido no contexto de limitações visuais. 
}
\subsection{Justificativa}

{


Existe uma ideia equivocada de que pessoas com deficiência possuem apenas necessidades relacionadas exclusivamente a sua deficiência, quando na verdade essas pessoas necessitam de tudo o que uma pessoa sem deficiência necessita somado as particularidades atreladas às suas limitações. A acessibilidade \textit{web} surge como um modificador de vida trazendo conforto e praticidade para os indivíduos. Porém esse conforto e praticidade não costuma ocorrer tão facilmente. Mesmos as pessoas com deficiência tendo voz, poder e direitos dentro da sociedade eles passam a ser escanteados pelas potenciais empresas que criam lojas \textit{on-line}. 

Por mais que a internet esteja difundida nos dias de hoje, as pessoas com deficiência, consumidores em potencial, acabam por serem esquecidos e negligenciados. Logo esse trabalho se faz relevante por apresentar opções para assegurar os direitos do público-alvo em questão e demonstrar a importância da construção de \textit{e-commerce} acessíveis, pois existe um nicho de mercado que está sendo deixado de lado e poderia ser uma vantagem competitiva às empresas que adotassem as técnicas de acessibilidade em seus sites de \textit{e-commerce}.
}
